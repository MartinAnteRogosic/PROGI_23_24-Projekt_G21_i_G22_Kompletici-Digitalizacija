\chapter{Arhitektura i dizajn sustava}
		


	
	Arhitektura aplikacije može se podijeliti na 3 djela. 
	
	\begin{packed_item}
		\item web poslužitelj
		
		\item web preglednik
		
		\item baza podataka
	
	\end{packed_item}
	
	\underbar{Web preglednik} je program koji korisniku omogućuje pristup web stranicama te multimedijskom sadržaju na njima. Web preglednik omogućuje korisniku pristup web poslužitelju putem HTTPS (eng. \textit{Hypertext Transfer Protokol Secure}) protokola te interpretira sadržaj kojeg je poslao Web poslužitelj u ljudski razumljiv oblik.
	
	\underbar{Web poslužitelj} je program koji prima i odgovara na korisnikove HTTPS zahtjeve. Web poslužitelj pokreće aplikaciju te korisniku šalje potrebne podatke kako bi korisnikov web preglednik mogao generirati frotnend aplikacije. Frontend predstavlja korisničko sučelje te je pisan u HTML-u, CSS-u i JavaScriptu uz pomoć React framework-a. Web poslužitelj je pisan u programskom jeziku Java i Spring Boot frameworku. Web poslužitelj komunicira sa bazom podataka te šalje podatke iz nje korisniku u prikladnom obliku.
	
	\underbar{Baza podataka} implementirana je putem PostgresSQL-a. Sadrži podatke potrebne za funkcioniranje aplikacije te njihove međusobne odnose. 

	
		

	
				
		\section{Baza podataka}
			
		
			
		Baza podataka za web aplikaciju koristi relacijski model podataka. Kao RDBMS (Relational Database Management System) koristi se PostgreSQL verzija 15.0. Koristeći ERDPlus, stvaramo ER dijagram baze podataka u kojem navodimo sve entitete, njihove atribute te međusobne veze. Daljnje tipove podataka, primarne, strane i alternativne ključeve navodimo u relacijskoj shemi baze podataka.
		
			
			\subsection{Opis tablica}

				\textbf{User}  Ovaj entitet sadržava sve važne informacije o korisniku aplikacije. Sadrži atribute: id, name, surname, function, password i email. Ovaj entitet u vezi je One-to-Many s entitetom Photos preko korisnikovog id-a te je u vezi s entitetom Document preko korisnikovog id-a.
		
				\begin{longtblr}[
					label=none,
					entry=none
					]{
						width = \textwidth,
						colspec={|X[6,l]|X[6, l]|X[20, l]|}, 
						rowhead = 1,
					} %definicija širine tablice, širine stupaca, poravnanje i broja redaka naslova tablice
					\hline \SetCell[c=3]{c}{\textbf{User}}	 \\ \hline[3pt]
					\SetCell{LightGreen}id & VARCHAR(10)	&  	Jedinstveni identifikator svakog zaposlenika  	\\ \hline
					name	& CHAR(30) &   Ime korisnika	\\ \hline
                    surname & CHAR(30) &   Prezime korisnika    \\ \hline 
					function & CHAR(10) &   Funkcija svakog zaposlenika (limitirano na Zaposlenik, Revizor, Računovođa i Direktor)   \\ \hline 
					password & VARCHAR(30)	&  	Korisnikova lozinka	\\ \hline 
					email & VARCHAR(50)   &   E-mail korisnika \\ \hline
				\end{longtblr}
				
				\textbf{Photos}  Ovaj entitet sadržava sve važne informacije o fotografijama pohranjenima u bazu. Sadrži atribute: photoID, url, imageName, uploadTime te strani ključ id preko korisnika. Ovaj entitet u vezi je One-to-Many s entitetom Photos preko korisnikovog id-a te je u vezi s entitetom Document preko korisnikovog id-a.

				\begin{longtblr}[
					label=none,
					entry=none
					]{
						width = \textwidth,
						colspec={|X[6,l]|X[6, l]|X[20, l]|}, 
						rowhead = 1,
					}
					\hline \SetCell[c=3]{c}{\textbf{Photos}}	 \\ \hline[3pt]
					\SetCell{LightBlue} id	& VARCHAR(10)   &   Jedinstveni identifikator zaposlenika izveden iz tablice RegisteredUser	\\ \hline
                    \SetCell{LightGreen} photoID  & VARCHAR(10)   &   Jedinstveni identifikator fotografije \\ \hline
                    url   & VARCHAR(255)   &  URL fotografije \\ \hline
                    imageName   & VARCHAR(10)  &  Ime prenesene fotografije \\ \hline
                    uploadTime  & TIMESTAMP  &  Datum i vrijeme prijenosa fotografije \\ \hline
				\end{longtblr}

				\textbf{Document}  Ovaj entitet sadržava sve važne informacije o dokumentima pohranjenima u bazu. Sadrži atribute: documentID, verifierID, correct, documentType, signed, verified, superVerified te strani ključ id preko korisnika. Ovaj entitet u vezi je One-to-One s entitetom Photos preko korisnikovog id-a, u Many-to-One vezi je s entitetom User te je u vezi One-to-Many s entitetom Archive preko korisnikovog id-a.

                \begin{longtblr}[
					label=none,
					entry=none
					]{
						width = \textwidth,
						colspec={|X[6,l]|X[6, l]|X[20, l]|}, 
						rowhead = 1,
					}
					\hline \SetCell[c=3]{c}{\textbf{Document}}	 \\ \hline[3pt]
                    \SetCell{LightBlue} id  &  VARCHAR(10)  &  Jedinstveni identifikator zaposlenika izveden iz tablice RegisteredUser	\\ \hline
                    \SetCell{LightGreen} documentID  &  VARCHAR(10)  &  Jedinstveni identifikator prenesenog dokumenta \\ \hline
                    verifierID  &  VARCHAR(10)  &  Jedinstveni identifikator zaposlenika koji je potvrdio dokument \\ \hline
                    correct  &  BOOLEAN  &  Potvrda ukoliko je rezultantni dokument ispravno skeniran \\ \hline
                    documentType  &  VARCHAR(20)  &  Tip dokumenta (limitirano na Račun, Ponuda i Interni dokument) \\ \hline
                    signed  &  BOOLEAN  &  Potvrda ukoliko je rezultantni dokument potpisan od strane direktora \\ \hline
                    verified  &  BOOLEAN  &  Potvrda ukoliko je rezultantni dokument potpisan od strane računovođe \\ \hline
                    toBeSigned  &  BOOLEAN  &  Potvrda ukoliko rezultantni dokument očekiva potpis od direktora\\ \hline
                \end{longtblr}

				\textbf{Archive}  Ovaj entitet sadržava sve važne informacije o arhiviranim dokumentima u bazi. Sadrži atribute: archiveID te strani ključ documentID preko entiteta Dokument. Ovaj entitet u vezi je One-to-One s entitetom Photos preko korisnikovog id-a, u Many-to-One vezi je s entitetom User te je u vezi One-to-Many s entitetom Archive preko korisnikovog id-a.

                \begin{longtblr}[
					label=none,
					entry=none
					]{
						width = \textwidth,
						colspec={|X[6,l]|X[6, l]|X[20, l]|}, 
						rowhead = 1,
					}
					\hline \SetCell[c=3]{c}{\textbf{Archive}}	 \\ \hline[3pt]
                    \SetCell{LightBlue} documentID  &  VARCHAR(10)  &  Jedinstveni identifikator dokumenta koji se arhivira \\ \hline
                    \SetCell{LightGreen} archiveId  &  VARCHAR(10)  &  Jedinstveni identifikator dokumenta koji se nalazi u arhivi računa \\ \hline
                \end{longtblr}

				\textbf{Statistics}  Ovaj entitet sadržava sve važne informacije o statistici svakog pojedinačnog zaposlenika. Sadrži atribute: timestampLogin, timestampLogout te strani ključ id preko korisnikovog identifikatora. Ovaj entitet je u vezi Many-to-One s entitetom User.

				\begin{longtblr}[
					label=none,
					entry=none
					]{
						width = \textwidth,
						colspec={|X[6,l]|X[6, l]|X[20, l]|}, 
						rowhead = 1,
					}
					\hline \SetCell[c=3]{c}{\textbf{Archive}}	 \\ \hline[3pt]
                    \SetCell{LightBlue} id  &  VARCHAR(10)  &  Jedinstveni identifikator zaposlenika čija se aktivnost na stranici sprema u statistiku \\ \hline
                    timestampLogin  &  TIMESTAMP  &  Datum i vrijeme login-a zaposlenika \\ \hline
					timestampLogout &  TIMESTAMP  &  Datum i vrijeme logout-a zaposlenika \\ \hline
                \end{longtblr}

			\subsection{Dijagram baze podataka}
				
				\begin{figure}[H]
					\includegraphics[scale=0.6]{slike/kompletici_v3_ER.PNG} %veličina slike u odnosu na originalnu datoteku i pozicija slike
					\centering
					\caption{ER dijagram baze podataka}
					\label{fig:promjene}
				\end{figure}
				
					\begin{figure}[H]
					\includegraphics[scale=0.5]{slike/kompletici_v3_REL.PNG} %veličina slike u odnosu na originalnu datoteku i pozicija slike
					\centering
					\caption{REL dijagram baze podataka}
					\label{fig:promjene}
				\end{figure}
				
				
			
			\eject
			
			
		\section{Dijagram razreda}
		
			\begin{figure}[H]
				\includegraphics[scale=0.5]{slike/dijagram_razred1.jpeg} %veličina slike u odnosu na originalnu datoteku i pozicija slike
				\centering
				\caption{Dijagram razreda 1}
				\label{fig:promjene}
			\end{figure}
			
			\begin{figure}[H]
				\includegraphics[scale=0.5]{slike/dijagram_razred2.jpeg} %veličina slike u odnosu na originalnu datoteku i pozicija slike
				\centering
				\caption{Dijagram razreda 2}
				\label{fig:promjene}
			\end{figure}
			
			
			
			\eject
		
		\iffalse
		\section{Dijagram stanja}
			
			
			\textbf{\textit{dio 2. revizije}}\\
			
			\textit{Potrebno je priložiti dijagram stanja i opisati ga. Dovoljan je jedan dijagram stanja koji prikazuje \textbf{značajan dio funkcionalnosti} sustava. Na primjer, stanja korisničkog sučelja i tijek korištenja neke ključne funkcionalnosti jesu značajan dio sustava, a registracija i prijava nisu. }
			
			Dijagram stanja prikazuje stanja objekta te prijelaze iz jednog stanja u drugo temeljene na dogadajima. Na slici je prikazan dijagram stanja za direktora. Nakon prijave, direktoru se prikazuje početna stranica na kojoj može pristupit svim svojim funkcionalnostima (skeniranju dokumenata, promjeni lozinke, pregledu zahtjeva itd.). Odabirom na skeniranje dokumenata direktoru se pruža mogućnost odabira upload-a dokumenta ili provjere već dostupnog dokumenta. Odabirom pregleda zahtjeva pružaju mu se opcije revizije, potpisivanja te arhiviranja dokumenta. Odabirom pregleda zapisa direktor može pregledati arhivu, povijest skeniranja te statistiku svih pojedinačnih zaposlenika. Direktoru se također pružaju opcije promjene lozinke, brisanja zaposlenika te objave dokumenta na društvenim mrežama (u ovom slučaju na društvenoj mreži "Facebook").
			
			\begin{figure}[H]
				\includegraphics[scale=0.5]{slike/dijagram_stanja_direktor.png} %veličina slike u odnosu na originalnu datoteku i pozicija slike
				\centering
				\caption{Dijagram stanja}
				\label{fig:promjene}
			\end{figure}

			\eject 
		
		\section{Dijagram aktivnosti}
			
			\textbf{\textit{dio 2. revizije}}\\
			
			 \textit{Potrebno je priložiti dijagram aktivnosti s pripadajućim opisom. Dijagram aktivnosti treba prikazivati značajan dio sustava.}
			
			Dijagram aktivnosti primjenjuje se za opis modela toka upravljanja ili toka podataka. Ne upotrebljava se za modeliranje događajima poticanog ponašanja. U modeliranju toka upravljanja svaki novi korak poduzima se nakon završenog prethodnog, a naglasak je na jednostavnosti. Na dijagramu aktivnosti prikazan je princip rada zaposlenika. Zaposlenik nakon uspješne prijave u sustav upload-a fotografije koje se daju na skeniranje, te aplikacija iste fotografije konvertira koristeći Tesseract.

			\begin{figure}[H]
				\includegraphics[scale=0.5]{slike/dijagram_aktivnosti.png} %veličina slike u odnosu na originalnu datoteku i pozicija slike
				\centering
				\caption{Dijagram aktivnosti}
				\label{fig:promjene}
			\end{figure}


			\eject
		\section{Dijagram komponenti}
		
			\textbf{\textit{dio 2. revizije}}\\
		
			 \textit{Potrebno je priložiti dijagram komponenti s pripadajućim opisom. Dijagram komponenti treba prikazivati strukturu cijele aplikacije.}

			 Dijagram komponenti prikazan na slici opisuje organizaciju i međuovisnost komponenti, interne strukture i odnose prema okolini. Sustavu se pristupa preko dva različita sučelja. Preko sučelja za dohvat HTML, CSS i JS datoteka poslužuju se datoteke koje pripadaju frontend dijelu aplikacije. Router je komponenta koja na upit s url određuje koja datoteka će se poslužiti na sučelje. Frontend dio se sastoji od niza JavaScript datoteka koje su raspoređene u logičke cjeline nazvane po tipovima aktora koji im pristupaju. Sve JavaScript datoteke ovise o React biblioteci iz koje dohvaćaju gotove komponente kao što su gumbi, forme i slično. Preko sučelja za dohvat JSON podataka pristupa se REST API komponenti. REST API poslužuje podatke koji pripadaju backend dijelu aplikacije. EntityFrameWorkCore je zadužen za dohvaćanje tablica iz baze podataka pomoću SQL upita. Podaci koji su pristigli iz baze se šalju dalje MVC arhitekturi u obliku DTO preko Services. React-view komponenta preko dostupnih sučelja komunicira sa aplikacijom te ovisno o korisnikovim akcijama osvježava prikaz i dohvaća nove podatke ili datoteke.

			 \begin{figure}[H]
				\includegraphics[scale=0.5]{slike/dijagram_komponenti.png} %veličina slike u odnosu na originalnu datoteku i pozicija slike
				\centering
				\caption{Dijagram komponenti}
				\label{fig:promjene}
			\end{figure}
			 
		\fi