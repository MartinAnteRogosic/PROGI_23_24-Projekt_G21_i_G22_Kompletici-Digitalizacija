%definira klasu dokumenta 
\documentclass[12pt]{report} 

%prostor izmedu naredbi \documentclass i \begin{document} se zove uvod. U njemu se nalaze naredbe koje se odnose na cijeli dokument
	
	%osnovni LaTex ne može riješiti sve probleme, pa se koriste različiti paketi koji olakšavaju izradu željenog dokumenta
	\usepackage[croatian]{babel} 
	\usepackage{amssymb}
	\usepackage{amsmath}
	\usepackage{txfonts}
	\usepackage{mathdots}
	\usepackage{titlesec}
	\usepackage{array}
	\usepackage{lastpage}
	\usepackage{etoolbox}
	\usepackage{tabularray}
	\usepackage{color, colortbl}
	\usepackage{adjustbox}
	\usepackage{geometry}
	\usepackage[classicReIm]{kpfonts}
	\usepackage{hyperref}
	\usepackage{fancyhdr}
	
	\usepackage{float}
	\usepackage{setspace}
	\restylefloat{table}
	
	
	\patchcmd{\chapter}{\thispagestyle{plain}}{\thispagestyle{fancy}}{}{} %redefiniranje stila stranice u paketu fancyhdr
	
	%oblik naslova poglavlja
	\titleformat{\chapter}{\normalfont\huge\bfseries}{\thechapter.}{20pt}{\Huge}
	\titlespacing{\chapter}{0pt}{0pt}{40pt}
	
	
	\linespread{1.3} %razmak između redaka
	
	\geometry{a4paper, left=1in, top=1in, right=1in}  %oblik stranice
	
	\hypersetup{ colorlinks, citecolor=black, filecolor=black, linkcolor=black,	urlcolor=black }   %izgled poveznice
	
	
	%prored smanjen između redaka u nabrajanjima i popisima
	\newenvironment{packed_enum}{
		\begin{enumerate}
			\setlength{\itemsep}{0pt}
			\setlength{\parskip}{0pt}
			\setlength{\parsep}{0pt}
		}{\end{enumerate}}
	
	\newenvironment{packed_item}{
		\begin{itemize}
			\setlength{\itemsep}{0pt}
			\setlength{\parskip}{0pt}
			\setlength{\parsep}{0pt}
		}{\end{itemize}}
	
	
	
	
	%boja za privatni i udaljeni kljuc u tablicama
	\definecolor{LightBlue}{rgb}{0.9,0.9,1}
	\definecolor{LightGreen}{rgb}{0.9,1,0.9}
	
	%Promjena teksta za dugačke tablice
	\DefTblrTemplate{contfoot-text}{normal}{Nastavljeno na idućoj stranici}
	\SetTblrTemplate{contfoot-text}{normal}
	\DefTblrTemplate{conthead-text}{normal}{(Nastavljeno)}
	\SetTblrTemplate{conthead-text}{normal}
	\DefTblrTemplate{middlehead,lasthead}{normal}{Nastavljeno od prethodne stranice}
	\SetTblrTemplate{middlehead,lasthead}{normal}
	
	%podesavanje zaglavlja i podnožja
	
	\pagestyle{fancy}
	\lhead{Programsko inženjerstvo}
	\rhead{Digitalizacija}
	\lfoot{Kompletići}
	\cfoot{stranica \thepage/\pageref{LastPage}}
	\rfoot{\today}
	\renewcommand{\headrulewidth}{0.2pt}
	\renewcommand{\footrulewidth}{0.2pt}
	
	
	\begin{document} 
		
		
		
		\begin{titlepage}
			\begin{center}
				\vspace*{\stretch{1.0}} %u kombinaciji s ostalim \vspace naredbama definira razmak između redaka teksta
				\LARGE Programsko inženjerstvo\\
				\large Ak. god. 2023./2024.\\
				
				\vspace*{\stretch{3.0}}
				
				\huge Digitalizacija\\
				\Large Dokumentacija, Rev. \textit{2}\\
				
				\vspace*{\stretch{12.0}}
				\normalsize
				Grupa: \textit{Kompletići}\\
				Voditelj: \textit{Martin Ante Rogošić}\\
				
				
				\vspace*{\stretch{1.0}}
				Datum predaje: \textit{19. 1. 2024.}\\
				
				\vspace*{\stretch{4.0}}
				
				Nastavnik: \textit{Igor Stančin}\\
				
			\end{center}
			
			
		\end{titlepage}
		
		
		\tableofcontents
		
		
		\chapter{Dnevnik promjena dokumentacije}
		
		\textbf{\textit{Kontinuirano osvježavanje}}\\
				
		
		\begin{longtblr}[
				label=none
			]{
				width = \textwidth, 
				colspec={|X[2]|X[13]|X[3]|X[3]|}, 
				rowhead = 1
			}
			\hline
			\textbf{Rev.}	& \textbf{Opis promjene/dodatka} & \textbf{Autori} & \textbf{Datum}\\[3pt] \hline
			
			0.1 & Napravljen predložak.	& Martin Ante Rogošić & 7.11.2023. 		\\[3pt] \hline 
			0.2	& Dodan opis projektnog zadatka & Martin Ante Rogošić & 9.11.2023. 	\\[3pt] \hline
			0.3 & Dodani svi obrasci uporabe. & Martin Ante Rogošić & 16.11.2023. \\[3pt] \hline
			0.4 & Dodani dijagrami i arhitektura sustava. & Luka Panđa, Emanuel Njegovec, Dominik Pavelić & 17.11.2023. \\[3pt] \hline
			1.0 & Gotova prva revizija dokumentacije. & Svi & 17.11.2023. \\[3pt] \hline
						
		\end{longtblr}
	
	
	

		\chapter{Opis projektnog zadatka}
		
		
		Cilj ovog projekta je razvoj i implementacija aplikacije Digitaliziraj, koja će tvrtkama olakšati digitalizaciju dokumenata nužnih za poslovanje koristeći OCR (eng. Optical character recognition). Aplikacija će olakšati digitalizaciju dokumenata unutar organizacija, te osigurati da svaki radnik organizacije dobije dokumente za koje je zadužen. Time, aplikacija će povećati učinkovitost organizacije te ubrzati njeno djelovanje.\\
		
		Neregistrirani korisnik mora napraviti korisnički račun kako bi mogao koristiti funkcionalnosti aplikacije. Za to su mu potrebni sljedeći podatci:
		
		\begin{packed_item}
			\item \textit{ime}
			\item \textit{prezime}
			\item \textit{email adresa}
			\item \textit{lozinka}
		\end{packed_item}
	
	Dodatno, ne registrirani korisnik mora odabrati na kojoj razini ovlasti želi stvoriti korisnički račun. Dostupne su 4 razine ovlasti:
	
	\begin{packed_item}
		\item \textit{zaposlenik}
		\item \textit{revizor}
		\item \textit{računovođa}
		\item \textit{direktor}
	\end{packed_item}
	
	Razne razine iznad su navedene od najniže do najviše. U većini slučajeva viša razina ovlasti ima sve funkcionalnosti onih ispod sebe, osim u onim slučajevima kada struktura tvrtke te razne odgovornosti zaposlenika to traže drugačije. Detalji o funkcionalnostima svake razine ovlasti nalaze se u nastavku dokumenta. Dodatno postoji još razina ovlasti direktor, koja predstavlja razinu ovlasti vlasnika tvrtke. Aplikacija sama stvara jedan korisnički račun s tom razinom ovlasti te su podatci potrebni za prijavu u taj korisnički račun dani vlasniku tvrtke.\\
	
	Zaposlenik ima najnižu razinu ovlasti unutar sustava. Korisnik s razinom ovlasti zaposlenik može učitavati slike u sustav te provjeriti je li sustav točno odradio konverziju slike u dokument. Jednom kad zaposlenik potvrdi da je pretvorba iz slike u dokument odrađena ispravno, generiran dokument se šalje jednom od revizora tvrtke na pregled. U svrhu ubrzanja procesa aplikacija omogućuje učitavanje do 50 slika istovremeno (ova funkcionalnost omogućena je i za sve više razine ovlasti). Pri tom zaposlenik i dalje mora svaku od konverzija potvrditi kao točnu prije nego što je dokument proslijeđen revizoru na pregled. Zaposlenik ima pristup povijesti svih dokumenata koje je skenirao.\\
	
	Revizor je druga najniža razina ovlasti unutar sustava. Posao revizora jest kontrola rada zaposlenika te preusmjeravanje dokumenata odgovarajućem računovođi (ovisno o vrsti dokumenta). Kao takav revizor ima pristup povijesti svih dokumenata nad kojima je vršio kontrolu. U slučaju da revizor skenira dokument aplikacija će automatski detektirati računovođu kojem vrsta skeniranog dokumenta treba biti poslana, te će nakon što revizor potvrdi ispravnost konverzije taj dokument biti automatski poslan odgovarajućem računovođi.\\
	
	
	Računovođa je najviša razina ovlasti u sustavu, izuzev direktora. Svaki računovođa je zadužen za obradu jedne od 3 vrste dokumenata koje aplikacija raspoznaje. To su računi, ponude i interni dokumenti. Računovođa ima pristup povijesti svih dokumenata one vrste za koju je zadužen. Posao računovođe je arhivirati dokumente. Pri arhiviranju dokumentu se dodjeljuje jedinstven broj arhiva. Računovođa također po potrebi može poslati dokument direktoru na potpisivanje. U tom slučaju direktor prima obavijest o tome. Nakon što direktor potpiše dokument računovođa dobiva obavijest da je dokument potpisan te ga onda može arhivirati. U slučaju da računovođa učita sliku dokumenta one vrste za koju nije zadužen aplikacija će mu automatski ponuditi slanje dokumenta odgovarajućem računovođi.\\
	
	Direktor je najviša razina ovlasti unutar sustava, te predstavlja najodgovorniju osoba unutar organizacije koja koristi aplikaciju za svoje poslovanje. Direktor ima uvid u povijest svih dokumenata te u statistike svih zaposlenika. U slučaju da mu računovođa pošalje zahtjev za potpis može potpisati dokument. U slučaju da direktor sam učitava slike u sustav te vrši konverziju, odmah mu se nudi mogućnost potpisivanja dokumenta te aplikacija automatski određuje računovođu kojem treba proslijediti učitan dokument. Dodatno za promotivne svrhe, direktoru je omogućena objava dokumenata na sljedećim društvenim mrežama: 
	
	\begin{packed_item}
		\item Facebook
		\item X
		\item Instagram
	\end{packed_item}
	
	\section{Potencijalna korist ovog projekta}
	
	Svojom strukturom i funkcionalnošću aplikacija ostvaruje brz i učinkovit sustav digitalizacije i distribucije dokumenata unutar organizacije. Njenom primjenom može se osigurati učinkovito poslovanje te raspodjela odgovornosti među zaposlenicima što pospješuje rad organizacije u mnogim aspektima. S obzirom na njenu općenitost aplikacija bi mogla biti  od interesa svim organizacijama koje se moraju baviti papirologijom. To uključuje: neprofitne humanitarne organizacije, tvrtke koje se natječu na tržištu, vladine agencije...\\
	
	\section{Slična rješenja}
	
	Na tržištu postoje razni sustavi za distribuciju dokumenata, također postoje i mnoge implementacije OCR-a. No vrlo je malen broj alata koji integriraju te dvije tehnologije u jedan sustav. Time se zaobilazi potreba da se odvojeni alati koriste za digitalizaciju dokumenata i njihovu distribuciju. \\
	
	\section{Mogućnost prilagodbe}
	
	Implementacija sustava u velikoj mjeri ovisi o strukturi organizacije i njenim potrebama za distribuciju dokumenata. Dodatno ako je klasifikacija dokumenata koju organizacija koristi složenija, potrebno je doraditi sustav automatskog prosljeđivanja dokumenata. Međutim implementacija dijelova sustava, koji se ne moraju mijenjati ovisno o potrebi klijenta, mogu se izvesti na način da je ta promjena relativno jednostavna.
	
	\section{Opseg projektnog zadatka}
	
	Konkretna implementacija izvedena u sklopu ovog projekta koristi relativno jednostavnu strukturu organizacije opisanu iznad, te razlikuje 3 različite vrste dokumenata i razina ovlasti. To je čini relativno jednostavnom za implementaciju no ujedno i lakom za integrirati u radni tok bilo koje organizacije koja koristi aplikaciju. 
	
	\section{Moguće nadogradnje}
	
	Mnoge nadogradnje su moguće na sustav. Integracija s alatima za preuređivanje digitalnih dokumenata, kako bi sustav postao znatno općenitije sredstvo za rukovanje dokumentima. Dodatno mogu se implementirati značajke koje bi omogućile komunikaciju među zaposlenicima,  generaciju raznih rasporeda i slično. Time bi aplikacija postala univerzalno sredstvo za upravljanje radom organizacije.
	

		
		
	
		\chapter{Specifikacija programske potpore}
		
	\section{Funkcionalni zahtjevi}
			
			\textbf{\textit{dio 1. revizije}}\\
			
			\textit{Navesti \textbf{dionike} koji imaju \textbf{interes u ovom sustavu} ili  \textbf{su nositelji odgovornosti}. To su prije svega korisnici, ali i administratori sustava, naručitelji, razvojni tim.}\\
				
			\textit{Navesti \textbf{aktore} koji izravno \textbf{koriste} ili \textbf{komuniciraju sa sustavom}. Oni mogu imati inicijatorsku ulogu, tj. započinju određene procese u sustavu ili samo sudioničku ulogu, tj. obavljaju određeni posao. Za svakog aktora navesti funkcionalne zahtjeve koji se na njega odnose.}\\
			
			
			\noindent \textbf{Dionici:}
			
			\begin{packed_enum}
				
				\item Neregistrirani korisnici
				\item Zaposlenici
				\item Revizori	
				\item Računovođe
				\item Direktor
				\item Razvojni tim
				
			\end{packed_enum}
			
			\noindent \textbf{Aktori i njihovi funkcionalni zahtjevi:}
			
			
			\begin{packed_enum}
				
				\item  \underbar{Neregistrirani korisnik (inicijator/sudionik) može:}
				\begin{packed_enum}
					
					\item Registrirati se u sustav
					
				\end{packed_enum}
			
				\item  \underbar{Zaposlenik (inicijator/sudionik) može:}
				
				\begin{packed_enum}
					
					\item Prijaviti se u sustav
					\item Učitati fotografiju u sustav te izvršiti konverziju te slike u dokument
					\item Učitati više fotografija od jednom u sustav te izvršiti njihovu konverziju u dokumente od jednom	
					\item Potvrditi ili odbiti točnost konverzije dokumenata u sustav
					\item Poslati dokument revizoru na pregled
					\item Pregledavati povijest svih dokumenata koje je zaposlenik unio u sustav
					
				\end{packed_enum}
			
				\item  \underbar{Revizor (inicijator/sudionik) može:}
				
				\begin{packed_enum}

					\item Provjeriti valjanost dokumenta kojeg mu je poslao zaposlenik
					\item Učitati fotografiju u sustav te izvršiti konverziju te slike u dokument 
					\item Učitati više fotografija od jednom u sustav te izvršiti njihovu konverziju u dokumente od jednom
					\item Potvrditi ili odbiti točnost konverzije dokumenata učitanih u sustav
					\item \textit{Aplikacija sama određuje kategoriju dokumenta no revizor to može ispraviti}
					\item Proslijediti dokument računovođi odgovornom za vrstu dokumenta kojoj dokument pripada
					\item Pregledati povijest svih dokumenata koje je revizor provjerio ili unio u sustav
					
				\end{packed_enum}
			
				\item  \underbar{Računovođa (inicijator/sudionik) može:}
				
				\begin{packed_enum}
					\item Računovođa može učitati jednu ili više slika u sustav te izvršiti njihovu konverziju u dokumente
					\item \textit{Automatsko prevođenje koje treba nekako pojasniti}
					\item Poslati dokumente direktoru na potpisivanje
					\item Arhivirati dokumente
					\item Pregledati povijest svih dokumenata one vrste za koju je računovođa zadužen
				\end{packed_enum}
			
				\item  \underbar{Direktor (inicijator/sudionik) može:}
			
				\begin{packed_enum}
					
					\item Potpisati dokument kojeg mu je poslao računovođa na potpisivanje
					\item Učitati jednu ili više fotografija u sustav te izvršiti njihovu konverziju u dokumente
					\item Potvrditi ili odbiti točnost konverzije dokumenata unesenih u sustav
					\item \textit{Automatizacija detekcije kategorije}
					\item Proslijediti dokument računovođi odgovornom za vrstu dokumenta kojoj dokument pripada
					
				\end{packed_enum}
			
			\end{packed_enum}
			
			\eject 
			
			
				
			\subsection{Obrasci uporabe}
					
				\subsubsection{Opis obrazaca uporabe}
					\textit{Funkcionalne zahtjeve razraditi u obliku obrazaca uporabe. Svaki obrazac je potrebno razraditi prema donjem predlošku. Ukoliko u nekom koraku može doći do odstupanja, potrebno je to odstupanje opisati i po mogućnosti ponuditi rješenje kojim bi se tijek obrasca vratio na osnovni tijek.}\\
					

					\noindent \underbar{\textbf{UC1 - Registracija}}
					\begin{packed_item}
	
						\item \textbf{Glavni sudionik:} Neregistrirani korisnik
						\item  \textbf{Cilj:} Registracija korisnika u sustav
						\item  \textbf{Sudionici:} Baza podataka
						\item  \textbf{Preduvjet:} -
						\item  \textbf{Opis osnovnog tijeka:}
						\item[] \begin{packed_enum}
	
							\item Neregistrirani korisnik odabire opciju za registraciju
							\item Neregistrirani korisnik unosi podatke za registraciju
							\item Korisnik prima obavijest o uspješnoj registraciji
							
						\end{packed_enum}
						
						\item  \textbf{Opis mogućih odstupanja:}
						\item[] \begin{packed_item}
	
							\item[2.a] Korisnik je odabrao već zauzetu ili neispravnu e-mail adresu, ili je dao ne dovoljno sigurnu lozinku
							\item[] \begin{packed_enum}
								
								\item Neregistriranog korisnika se vraća na stranicu za registraciju te ga se obavještava o neuspješnoj registraciji
								\item Korisnik mijenja nepravilne podatke ili odustaje od registracije	
							\end{packed_enum}
						\end{packed_item}
					\end{packed_item}
					
					
					\noindent \underbar{\textbf{UC2 - Prijava u sustav}}
					\begin{packed_item}
						
						\item \textbf{Glavni sudionik: } Korisnik sustava
						\item  \textbf{Cilj:} Prijava u sustav te pristup korisničkom sučelju
						\item  \textbf{Sudionici:} Baza podataka
						\item  \textbf{Preduvjet:}Registracija
						\item  \textbf{Opis osnovnog tijeka:}
						
						\item[] \begin{packed_enum}
							
							\item Korisnik odabire opciju za prijavu
							\item Korisnik unosi potrebne korisničke podatke
							\item Korisnik dobiva pristup korisničkom sučelju
						\end{packed_enum}
						
						\item  \textbf{Opis mogućih odstupanja:}
						
						\item[] \begin{packed_item}
							
							\item[2.a] Korisnik je upisao nepostojeće ili pogrešne korisničke podatke
							\item[] \begin{packed_enum}
								\item Korisnika se vraća na stranicu za prijavu te ga se obavještava o neuspjeloj prijavi
							\end{packed_enum}	
						\end{packed_item}
					\end{packed_item}
				
				\noindent \underbar{\textbf{UC3 -Promjena lozinke}}
				\begin{packed_item}
					
					\item \textbf{Glavni sudionik: } Korisnik
					\item  \textbf{Cilj:} Promjena lozinke koja se koristi pri prijavi u sustav
					\item  \textbf{Sudionici:} Baza podataka
					\item  \textbf{Preduvjet:} Registracija
					\item  \textbf{Opis osnovnog tijeka:}
					
					\item[] \begin{packed_enum}
						
						\item Korisnik bira opciju za promjenu lozinke
						\item Unosi staru lozinku kao potvrdu svog identiteta
						\item Bira novu lozinku te je upisuje dva puta
						\item Korisnik je obaviješten o uspješnoj promjeni lozink
					\end{packed_enum}
					
					\item  \textbf{Opis mogućih odstupanja:}
					
					\item[] \begin{packed_item}
						
						\item[2.a] Korisnik je unio pogrešnu staru lozinku
						\item[] \begin{packed_enum}
							
							\item Korisnika se obavještava o pogrešnom unosu lozinke te ga se vraća na stranicu za promjenu lozinke
							\item Korisnik ponovno unosi staru lozinku ili odustaje od promjene lozinke
							
						\end{packed_enum}
					\end{packed_item}
				\end{packed_item}
				
				
				\noindent \underbar{\textbf{UC4 - Skeniranje jedne fotografije}}
				\begin{packed_item}
					
					\item \textbf{Glavni sudionik: }Korisnik
					\item  \textbf{Cilj:} Unijeti fotografiju u sustav te izvršiti njenu konverziju u dokument
					\item  \textbf{Sudionici:} Baza podataka
					\item  \textbf{Preduvjet:} Prijava u sustav
					\item  \textbf{Opis osnovnog tijeka:}
					
					\item[] \begin{packed_enum}
						
						\item Korisnik odabire opciju za učitavanje jedne fotografije
						\item Korisnik odabire fotografiju iz datotečnog sustava svog računala
						\item Korisnik dobiva obavijest o provedenoj konverziji
						
					\end{packed_enum}
					
					\item  \textbf{Opis mogućih odstupanja:}
					
					\item[] \begin{packed_item}
						
						\item[2.a] Korisnik je odabrao datoteku koja ne postoji na njegovom računalu
						\item[] \begin{packed_enum}
							
							\item Sustav obavještava korisnika o nastaloj pogrešci te ga preusmjerava na korisničko sučelje
							
						\end{packed_enum}
					\end{packed_item}
				\end{packed_item}
				
				\noindent \underbar{\textbf{UC5 -Skeniranje više fotografija}}
				\begin{packed_item}
					
					\item \textbf{Glavni sudionik: }Korisnik
					\item  \textbf{Cilj:} Unijeti više fotografija u sustav te izvršiti njihovu konverziju u dokumente
					\item  \textbf{Sudionici:} Baza podataka
					\item  \textbf{Preduvjet:} Prijava u sustav
					\item  \textbf{Opis osnovnog tijeka:}
					
					\item[] \begin{packed_enum}
						
						\item Korisnik odabire opciju za učitavanje više dokumenata
						\item Korisnik odabire više datoteka iz svog datotečnog sustava
						\item Korisnik dobiva obavijest o provedenim konverzijama
					
					\end{packed_enum}
					
					\item  \textbf{Opis mogućih odstupanja:}
					
					\item[] \begin{packed_item}
						
						\item[2.a] Jedna od fotografija koje je korisnik unio ne postoji u datotečnom sustavu
						\item[] \begin{packed_enum}
							
							\item Sustav provodi konverziju pronađenih dokumenata
							\item Sustav šalje korisniku obavijest o ne pronađenim dokumentima
							
						\end{packed_enum}
					\end{packed_item}
				\end{packed_item}
			
			
				\noindent \underbar{\textbf{UC6 -Potvrda ispravnosti konverzije}}
				\begin{packed_item}
					
					\item \textbf{Glavni sudionik: }
					\item  \textbf{Cilj:} Potvrditi točnost konverzije dokumenta unesenih u sustav
					\item  \textbf{Sudionici:} Baza podataka
					\item  \textbf{Preduvjet:} Korisnik je prijavljen te je učitao jednu ili više fotografija u sustav
					\item  \textbf{Opis osnovnog tijeka:}
					
					\item[] \begin{packed_enum}
						
						\item Korisnik odabire opciju za potvrdu ispravnosti konverzije
						\item Korisniku se prikazuju fotografija i dokument generiran iz slike paralelno na ekranu, te mu se nudi opcija za prihvaćanje ili odbijanje točnosti konverzije
						 
						
					\end{packed_enum}
				\end{packed_item}
			
				\noindent \underbar{\textbf{UC7 -Prosljeđivanje dokumenata}}
				\begin{packed_item}
					
					\item \textbf{Glavni sudionik: }Korisnik
					\item \textbf{Cilj:} Proslijediti dokument potvrđene točnosti drugom korisniku s prikladnom razinom ovlasti za daljnju obradu dokumenta
					\item \textbf{Sudionici:} Baza podataka i drugi korisnici s višom razinom ovlasti
					\item \textbf{Preduvjet:} Korisnik je unio u sustav fotografiju, izvršio konverziju i potvrdio njenu točnost
					\item \textbf{Opis osnovnog tijeka:}
					
					\item[] \begin{packed_enum}
						
						\item Korisnik odabire opciju za slanje dokumenta
						\item Sustav prosljeđuje dokument odgovarajućem korisniku
						\item Korisnik dobiva obavijest to proslijeđenosti dokumenta
					\end{packed_enum}
				\end{packed_item}
			
				\noindent \underbar{\textbf{UC8 -Pregled vlastite povijesti}}
				\begin{packed_item}
					
					\item \textbf{Glavni sudionik: }Zaposlenik
					\item  \textbf{Cilj:} Pregled svih dokumenata koje je zaposlenik unio u sustav
					\item  \textbf{Sudionici:} Baza podataka
					\item  \textbf{Preduvjet:} Prijava
					\item  \textbf{Opis osnovnog tijeka:}
					
					\item[] \begin{packed_enum}
						
						\item Zaposlenik odabire opciju za pregled povijesti
						\item Aplikacija prikazuje korisniku listu svih dokumenata koje je unio u sustav
						\item Zaposlenik odabire dokument iz liste
						\item Aplikacija korisniku prikazuje dokument i sliku iz koje je dokument generiran
						
					\end{packed_enum}
				\end{packed_item}
			
			
		
			\noindent \underbar{\textbf{UC9 -Verifikacija točnosti konverzije}}
			\begin{packed_item}
				
				\item \textbf{Glavni sudionik: } Korisnik
				\item  \textbf{Cilj:} Potvrditi točnost generiranog dokumenta temeljem slike iz koje je dokument generiran
				\item  \textbf{Sudionici:} Baza podataka
				\item  \textbf{Preduvjet:} Skeniranje jedne ili više fotografija
				\item  \textbf{Opis osnovnog tijeka:}
				
				\item[] \begin{packed_enum}
					
					\item Korisnik odabire opciju za potvrdu konverzije
					\item Aplikacija korisniku prikazuje listu dokumenata čiju je točnost konverzije potrebno potvrditi
					\item Korisnik odabire dokument s liste, te mu aplikacija prikazuje dokument i sliku iz koje je dokument generiran. Dodatno aplikacija nudi mogućnost potvrde ili odbijanja točnosti konverzije
					\item Odabir korisnika o točnosti (ili netočnosti) konverzije pohranjuje se u bazu podataka
					
				\end{packed_enum}
			\end{packed_item}
		
			\noindent \underbar{\textbf{UC10 -Automatska kategorizacija dokumenata}}
			\begin{packed_item}
				
				\item \textbf{Glavni sudionik: }Revizor ili Računovođa ili Direktor
				\item  \textbf{Cilj:} Automatsko prepoznavanje vrste dokumenta pri konverziji dokumenta iz slike
				\item  \textbf{Sudionici:} Baza podataka
				\item  \textbf{Preduvjet:} Skeniranje jedne ili više fotografija
				\item  \textbf{Opis osnovnog tijeka:}
				
				\item[] \begin{packed_enum}
					
					\item Revizor ili Računovođa ili Direktor unose jednu ili više fotografija u sustav
					\item Sustav nakon konverzije automatski određuje kategoriju dokumenta
					\item Pri verifikaciji točnosti konverzije korisnicima se omogućuje i verifikacija točnosti automatske kategorizacije
					\item Nakon što je točnost automatske kategorizacije dokumenta potvrđena, kategorija dokumenta pohranjuje se u bazu podataka
				\end{packed_enum}
			\end{packed_item}
			
			\noindent \underbar{\textbf{UC11 -Slanje dokumenata na potpis}}
			\begin{packed_item}
				
				\item \textbf{Glavni sudionik: } Računovođa
				\item  \textbf{Cilj:} Slanje dokumenta Direktoru na potpis
				\item  \textbf{Sudionici:} Baza podataka, Direktor
				\item  \textbf{Preduvjet:} Prisutnost jednog ili više dokumenata koje treba obraditi u sustavu
				\item  \textbf{Opis osnovnog tijeka:}
				
				\item[] \begin{packed_enum}
					
					\item Računovođa bira jedan dokument iz liste dokumenata koje treba obraditi 
					\item Računovođa odabire opciju za slanje dokumenta za potpis
					\item Dokument je privremeno uklonjen iz liste dokumenata koje treba obraditi dok direktor ne potpiše dokument
				\end{packed_enum}
			\end{packed_item}
		
			\noindent \underbar{\textbf{UC12 -Arhiviranje}}
			\begin{packed_item}
				
				\item \textbf{Glavni sudionik: }Računovođa
				\item  \textbf{Cilj:} Arhiviranje dokumenata
				\item  \textbf{Sudionici:} Baza podataka
				\item  \textbf{Preduvjet:} Prisutnost jednog ili više dokumenata koji su spremni za arhiviranje
				\item  \textbf{Opis osnovnog tijeka:}
				
				\item[] \begin{packed_enum}
					
					\item Računovođa bira jedan ili više dokumenata iz liste dokumenata koje treba obraditi 
					\item Aplikacija nudi mogućnost za arhiviranje jednog ili više dokumenata
					\item Odabirom opcije za arhiviranje svi odabrani dokumenti dobivaju jedinstven broj arhiva
					\item Baza podataka pohranjuje jedinstven broj arhiva za svaki arhiviran dokument
					
				\end{packed_enum}
			\end{packed_item}
			
			\noindent \underbar{\textbf{UC$<$broj obrasca$>$ -$<$ime obrasca$>$}}
			\begin{packed_item}
				
				\item \textbf{Glavni sudionik: }$<$sudionik$>$
				\item  \textbf{Cilj:} $<$cilj$>$
				\item  \textbf{Sudionici:} $<$sudionici$>$
				\item  \textbf{Preduvjet:} $<$preduvjet$>$
				\item  \textbf{Opis osnovnog tijeka:}
				
				\item[] \begin{packed_enum}
					
					\item $<$opis korak jedan$>$
					\item $<$opis korak dva$>$
					\item $<$opis korak tri$>$
					\item $<$opis korak četiri$>$
					\item $<$opis korak pet$>$
				\end{packed_enum}
				
				\item  \textbf{Opis mogućih odstupanja:}
				
				\item[] \begin{packed_item}
					
					\item[2.a] $<$opis mogućeg scenarija odstupanja u koraku 2$>$
					\item[] \begin{packed_enum}
						
						\item $<$opis rješenja mogućeg scenarija korak 1$>$
						\item $<$opis rješenja mogućeg scenarija korak 2$>$
						
					\end{packed_enum}
					\item[2.b] $<$opis mogućeg scenarija odstupanja u koraku 2$>$
					\item[3.a] $<$opis mogućeg scenarija odstupanja  u koraku 3$>$
					
				\end{packed_item}
			\end{packed_item}
		
			\noindent \underbar{\textbf{UC13 -Pregled svih dokumenata određene kategorije}}
			\begin{packed_item}
				
				\item \textbf{Glavni sudionik: }Računovođa
				\item  \textbf{Cilj:} Pregled svih dokumenata određene kategorije koje su svi korisnici unijeli u sustav
				\item  \textbf{Sudionici:} Baza podataka
				\item  \textbf{Preduvjet:} Prisutnost jednog ili više dokumenata pretraživane kategorije u sustavu
				\item  \textbf{Opis osnovnog tijeka:}
				
				\item[] \begin{packed_enum}
					
					\item Računovođa odabire opciju za pregled povijesti svih dokumenata određene kategorije
					\item Aplikacije prikazuje računovođi listu svih dokumenata određene kategorije u sustavu
					\item Računovođa odabire dokument iz liste
					\item Aplikacija prikazuje računovođi dokument i sliku iz koje je dokument generiran
					\item $<$opis korak pet$>$
				\end{packed_enum}
			\end{packed_item}
			
			\noindent \underbar{\textbf{UC14 -Potpisivanje dokumenta}}
			\begin{packed_item}
				
				\item \textbf{Glavni sudionik: }Direktor
				\item  \textbf{Cilj:} Potpisivanje određenog dokumenta
				\item  \textbf{Sudionici:} Baza podataka
				\item  \textbf{Preduvjet:} Prisutnost jednog ili više dokumenata u sustavu za koje je zatražen potpis
				\item  \textbf{Opis osnovnog tijeka:}
				
				\item[] \begin{packed_enum}
					
					\item Direktor odabire opciju za potpisivanje dokumenata
					\item Aplikacija direktoru prikazuje listu dokumenata za koje je zatražen potpis
					\item Odabirom dokumenta aplikacija nudi direktoru opciju za potpis
					\item Direktor potpisuje dokument ili odbija potpisati dokument
				\end{packed_enum}
			\end{packed_item}
			
			\noindent \underbar{\textbf{UC15 - Promjena kategorije dokumenta}}
			\begin{packed_item}
				
				\item \textbf{Glavni sudionik: } Računovođa, Direktor, Revizor
				\item  \textbf{Cilj:} Promjena kategorije dokumenta kojeg su u sustav unijeli računovođa ili direktor
				\item  \textbf{Sudionici:} Baza podataka
				\item  \textbf{Preduvjet:} Prisutnost jednog ili više dokumenata u sustavu čija je ispravnost konverzije ispravna to automatska kategorizacija nije
				\item  \textbf{Opis osnovnog tijeka:}
				
				\item[] \begin{packed_enum}
					
					\item Direktor, računovođa ili revizor odabiru opciju za potvrdu ispravnosti automatske kategorizacije dokumenta
					\item Aplikacija prikazuje fotografija i dokument generiran iz fotografije te mu se nudi opcija za potvrdu točnosti ili promjenu kategorije dokumenta
					\item Baza podataka pohranjuje odabir korisnika
				\end{packed_enum}
			\end{packed_item}
			
			\noindent \underbar{\textbf{UC16 -Objava dokumenata na društvenoj mreži}}
			\begin{packed_item}
				
				\item \textbf{Glavni sudionik: } Direktor
				\item  \textbf{Cilj:} Objava dokumenata na društvenoj mreži
				\item  \textbf{Sudionici:} Baza podataka
				\item  \textbf{Preduvjet:} Prisutnost dokumenata u sustavu
				\item  \textbf{Opis osnovnog tijeka:}
				
				\item[] \begin{packed_enum}
					
					\item Direktor odabire opciju za objavu dokumenta na društvenoj mreži
					\item Direktor odabire na kojoj od ponuđenih društvenih mreža želi objaviti dokument
					\item Direktor unosi potrebne podatke za objavu dokumenta koje traži specifična društvena mreža
					\item Dokument se objavljuje na društvenoj mreži
					
				\end{packed_enum}
			\end{packed_item}
		
			\noindent \underbar{\textbf{UC17 -Pregled povijesti svih dokumenata}}
			\begin{packed_item}
				
				\item \textbf{Glavni sudionik: }Direktor
				\item  \textbf{Cilj:} Pregled povijesti svih dokumenata
				\item  \textbf{Sudionici:} Baza podataka
				\item  \textbf{Preduvjet:} Prisutnost dokumenata u sustavu
				\item  \textbf{Opis osnovnog tijeka:}
				
				\item[] \begin{packed_enum}
					
					\item Direktor odabire opciju za pregled povijesti dokumenata
					\item Aplikacija direktoru prikazuje listu svih dokumenata ikad unesenih u sustav
					\item Odabirom dokumenta sa liste prikazuje se fotografija i dokument generiran iz fotografije
				
				\end{packed_enum}
			\end{packed_item}
		
			\noindent \underbar{\textbf{UC18 -Brisanje dokumenta iz Arhiva}}
			\begin{packed_item}
				
				\item \textbf{Glavni sudionik: }Direktor
				\item  \textbf{Cilj:}Brisanje dokumenta iz Arhiva
				\item  \textbf{Sudionici:} Baza podataka
				\item  \textbf{Preduvjet:} Prisutnost jednog ili više arhiviranih dokumenata u sustavu
				\item  \textbf{Opis osnovnog tijeka:}
				
				\item[] \begin{packed_enum}
					
					\item Direktor odabire opciju za brisanje dokumenta iz arhiva
					\item Aplikacija prikazuje listu svih arhiviranih dokumenata, te direktor odabire onog kojeg želi učitati
					\item Direktor unosi svoju lozinku kao potvrdu svog identiteta i odluke
					\item Baza briše dokument iz sustava
					
				\end{packed_enum}
			\end{packed_item}
		
			\noindent \underbar{\textbf{UC19 -Pregled statistike zaposlenika}}
			\begin{packed_item}
				
				\item \textbf{Glavni sudionik: } Direktor 
				\item  \textbf{Cilj:} Pregled statistike zaposlenika
				\item  \textbf{Sudionici:} Baza podataka, korisnici
				\item  \textbf{Preduvjet:} -
				\item  \textbf{Opis osnovnog tijeka:}
				
				\item[] \begin{packed_enum}
					
					\item Direktor odabire opciju za pregled statistike zaposlenika
					\item Aplikacija nudi listu svih korisnika
					\item Direktor odabire zaposlenika čije statistike želi pogledati
					\item Aplikacija prikazuje tražene podatke
				\end{packed_enum}
			\end{packed_item}
			
			\noindent \underbar{\textbf{UC20 -Promjena razine ovlasti}}
			\begin{packed_item}
				
				\item \textbf{Glavni sudionik: }Direktor 
				\item  \textbf{Cilj:} Promjena razine ovlasti korisnika
				\item  \textbf{Sudionici:} Baza podataka, korisnici
				\item  \textbf{Preduvjet:} -
				\item  \textbf{Opis osnovnog tijeka:}
				
				\item[] \begin{packed_enum}
					
					\item Direktor odabire opciju za promjenu razine ovlasti korisnika
					\item Sustav prikazuje direktoru listu svih korisnika sustava
					\item Direktor odabire zaposlenika te mu dodjeljuje novu razinu ovlasti
					\item Baza podataka sprema unesene promjene
					
				\end{packed_enum}
			\end{packed_item}
			
			\noindent \underbar{\textbf{UC21 -Brisanje korisničkog računa}}
			\begin{packed_item}
				
				\item \textbf{Glavni sudionik: } Korisnik
				\item  \textbf{Cilj:} Brisanje korisničkog računa
				\item  \textbf{Sudionici:} Baza podataka
				\item  \textbf{Preduvjet:} Registracija
				\item  \textbf{Opis osnovnog tijeka:}
				
				\item[] \begin{packed_enum}
					
					\item Korisnik bira opciju za brisanje korisničkog računa
					\item Korisnik unosi svoju korisničku lozinku kao potvrdu svog odabira i identiteta
					\item Aplikacija korisnika preusmjerava na stranicu za prijavu
					\item Baza podataka briše korisnički račun
				\end{packed_enum}
			\end{packed_item}
				
					
				\subsubsection{Dijagrami obrazaca uporabe}
					
					\textit{Prikazati odnos aktora i obrazaca uporabe odgovarajućim UML dijagramom. Nije nužno nacrtati sve na jednom dijagramu. Modelirati po razinama apstrakcije i skupovima srodnih funkcionalnosti.}
				\eject		
				
			\subsection{Sekvencijski dijagrami}
				
				\textbf{\textit{dio 1. revizije}}\\
				
				\textit{Nacrtati sekvencijske dijagrame koji modeliraju najvažnije dijelove sustava (max. 4 dijagrama). Ukoliko postoji nedoumica oko odabira, razjasniti s asistentom. Uz svaki dijagram napisati detaljni opis dijagrama.}
				\eject
	
		\section{Ostali zahtjevi}
		
			\textbf{\textit{dio 1. revizije}}\\
		 
			 \textit{Nefunkcionalni zahtjevi i zahtjevi domene primjene dopunjuju funkcionalne zahtjeve. Oni opisuju \textbf{kako se sustav treba ponašati} i koja \textbf{ograničenja} treba poštivati (performanse, korisničko iskustvo, pouzdanost, standardi kvalitete, sigurnost...). Primjeri takvih zahtjeva u Vašem projektu mogu biti: podržani jezici korisničkog sučelja, vrijeme odziva, najveći mogući podržani broj korisnika, podržane web/mobilne platforme, razina zaštite (protokoli komunikacije, kriptiranje...)... Svaki takav zahtjev potrebno je navesti u jednoj ili dvije rečenice.}
			 
			 
			 
	
		\chapter{Arhitektura i dizajn sustava}
		
		\textbf{\textit{dio 1. revizije}}\\

		\textit{ Potrebno je opisati stil arhitekture te identificirati: podsustave, preslikavanje na radnu platformu, spremišta podataka, mrežne protokole, globalni upravljački tok i sklopovsko-programske zahtjeve. Po točkama razraditi i popratiti odgovarajućim skicama:}
	\begin{itemize}
		\item 	\textit{izbor arhitekture temeljem principa oblikovanja pokazanih na predavanjima (objasniti zašto ste baš odabrali takvu arhitekturu)}
		\item 	\textit{organizaciju sustava s najviše razine apstrakcije (npr. klijent-poslužitelj, baza podataka, datotečni sustav, grafičko sučelje)}
		\item 	\textit{organizaciju aplikacije (npr. slojevi frontend i backend, MVC arhitektura) }		
	\end{itemize}

	
		

		

				
		\section{Baza podataka}
			
			\textbf{\textit{dio 1. revizije}}\\
			
		\textit{Potrebno je opisati koju vrstu i implementaciju baze podataka ste odabrali, glavne komponente od kojih se sastoji i slično.}
		
			\subsection{Opis tablica}
			

				\textit{Svaku tablicu je potrebno opisati po zadanom predlošku. Lijevo se nalazi točno ime varijable u bazi podataka, u sredini se nalazi tip podataka, a desno se nalazi opis varijable. Svjetlozelenom bojom označite primarni ključ. Svjetlo plavom označite strani ključ}
				
				
				\begin{longtblr}[
					label=none,
					entry=none
					]{
						width = \textwidth,
						colspec={|X[6,l]|X[6, l]|X[20, l]|}, 
						rowhead = 1,
					} %definicija širine tablice, širine stupaca, poravnanje i broja redaka naslova tablice
					\hline \SetCell[c=3]{c}{\textbf{korisnik - ime tablice}}	 \\ \hline[3pt]
					\SetCell{LightGreen}IDKorisnik & INT	&  	Lorem ipsum dolor sit amet, consectetur adipiscing elit, sed do eiusmod  	\\ \hline
					korisnickoIme	& VARCHAR &   	\\ \hline 
					email & VARCHAR &   \\ \hline 
					ime & VARCHAR	&  		\\ \hline 
					\SetCell{LightBlue} primjer	& VARCHAR &   	\\ \hline 
				\end{longtblr}
				
				
			
			\subsection{Dijagram baze podataka}
				\textit{ U ovom potpoglavlju potrebno je umetnuti dijagram baze podataka. Primarni i strani ključevi moraju biti označeni, a tablice povezane. Bazu podataka je potrebno normalizirati. Podsjetite se kolegija "Baze podataka".}
			
			\eject
			
			
		\section{Dijagram razreda}
		
			\textit{Potrebno je priložiti dijagram razreda s pripadajućim opisom. Zbog preglednosti je moguće dijagram razlomiti na više njih, ali moraju biti grupirani prema sličnim razinama apstrakcije i srodnim funkcionalnostima.}\\
			
			\textbf{\textit{dio 1. revizije}}\\
			
			\textit{Prilikom prve predaje projekta, potrebno je priložiti potpuno razrađen dijagram razreda vezan uz \textbf{generičku funkcionalnost} sustava. Ostale funkcionalnosti trebaju biti idejno razrađene u dijagramu sa sljedećim komponentama: nazivi razreda, nazivi metoda i vrste pristupa metodama (npr. javni, zaštićeni), nazivi atributa razreda, veze i odnosi između razreda.}\\
			
			\textbf{\textit{dio 2. revizije}}\\			
			
			\textit{Prilikom druge predaje projekta dijagram razreda i opisi moraju odgovarati stvarnom stanju implementacije}
			
			
			
			\eject
		
		\section{Dijagram stanja}
			
			
			\textbf{\textit{dio 2. revizije}}\\
			
			\textit{Potrebno je priložiti dijagram stanja i opisati ga. Dovoljan je jedan dijagram stanja koji prikazuje \textbf{značajan dio funkcionalnosti} sustava. Na primjer, stanja korisničkog sučelja i tijek korištenja neke ključne funkcionalnosti jesu značajan dio sustava, a registracija i prijava nisu. }
			
			
			\eject 
		
		\section{Dijagram aktivnosti}
			
			\textbf{\textit{dio 2. revizije}}\\
			
			 \textit{Potrebno je priložiti dijagram aktivnosti s pripadajućim opisom. Dijagram aktivnosti treba prikazivati značajan dio sustava.}
			
			\eject
		\section{Dijagram komponenti}
		
			\textbf{\textit{dio 2. revizije}}\\
		
			 \textit{Potrebno je priložiti dijagram komponenti s pripadajućim opisom. Dijagram komponenti treba prikazivati strukturu cijele aplikacije.}

		\chapter{Implementacija i korisničko sučelje}
		
		
		\section{Korištene tehnologije i alati}
		
			\textbf{\textit{dio 2. revizije}}
			
			 \textit{Detaljno navesti sve tehnologije i alate koji su primijenjeni pri izradi dokumentacije i aplikacije. Ukratko ih opisati, te navesti njihovo značenje i mjesto primjene. Za svaki navedeni alat i tehnologiju je potrebno \textbf{navesti internet poveznicu} gdje se mogu preuzeti ili više saznati o njima}.
			
			
			\eject 
		
	
		\section{Ispitivanje programskog rješenja}
			
			\textbf{\textit{dio 2. revizije}}\\
			
			 \textit{U ovom poglavlju je potrebno opisati provedbu ispitivanja implementiranih funkcionalnosti na razini komponenti i na razini cijelog sustava s prikazom odabranih ispitnih slučajeva. Studenti trebaju ispitati temeljnu funkcionalnost i rubne uvjete.}
	
			
			\subsection{Ispitivanje komponenti}
			\textit{Potrebno je provesti ispitivanje jedinica (engl. unit testing) nad razredima koji implementiraju temeljne funkcionalnosti. Razraditi \textbf{minimalno 6 ispitnih slučajeva} u kojima će se ispitati redovni slučajevi, rubni uvjeti te izazivanje pogreške (engl. exception throwing). Poželjno je stvoriti i ispitni slučaj koji koristi funkcionalnosti koje nisu implementirane. Potrebno je priložiti izvorni kôd svih ispitnih slučajeva te prikaz rezultata izvođenja ispita u razvojnom okruženju (prolaz/pad ispita). }
			
			
			
			\subsection{Ispitivanje sustava}
			
			 \textit{Potrebno je provesti i opisati ispitivanje sustava koristeći radni okvir Selenium\footnote{\url{https://www.seleniumhq.org/}}. Razraditi \textbf{minimalno 4 ispitna slučaja} u kojima će se ispitati redovni slučajevi, rubni uvjeti te poziv funkcionalnosti koja nije implementirana/izaziva pogrešku kako bi se vidjelo na koji način sustav reagira kada nešto nije u potpunosti ostvareno. Ispitni slučaj se treba sastojati od ulaza (npr. korisničko ime i lozinka), očekivanog izlaza ili rezultata, koraka ispitivanja i dobivenog izlaza ili rezultata.\\ }
			 
			 \textit{Izradu ispitnih slučajeva pomoću radnog okvira Selenium moguće je provesti pomoću jednog od sljedeća dva alata:}
			 \begin{itemize}
			 	\item \textit{dodatak za preglednik \textbf{Selenium IDE} - snimanje korisnikovih akcija radi automatskog ponavljanja ispita	}
			 	\item \textit{\textbf{Selenium WebDriver} - podrška za pisanje ispita u jezicima Java, C\#, PHP koristeći posebno programsko sučelje.}
			 \end{itemize}
		 	\textit{Detalji o korištenju alata Selenium bit će prikazani na posebnom predavanju tijekom semestra.}
			
			\eject 
		
		
		\section{Dijagram razmještaja}
			
			\textbf{\textit{dio 2. revizije}}
			
			 \textit{Potrebno je umetnuti \textbf{specifikacijski} dijagram razmještaja i opisati ga. Moguće je umjesto specifikacijskog dijagrama razmještaja umetnuti dijagram razmještaja instanci, pod uvjetom da taj dijagram bolje opisuje neki važniji dio sustava.}
			
			\eject 
		
		\section{Upute za puštanje u pogon}
		
			\textbf{\textit{dio 2. revizije}}\\
		
			 \textit{U ovom poglavlju potrebno je dati upute za puštanje u pogon (engl. deployment) ostvarene aplikacije. Na primjer, za web aplikacije, opisati postupak kojim se od izvornog kôda dolazi do potpuno postavljene baze podataka i poslužitelja koji odgovara na upite korisnika. Za mobilnu aplikaciju, postupak kojim se aplikacija izgradi, te postavi na neku od trgovina. Za stolnu (engl. desktop) aplikaciju, postupak kojim se aplikacija instalira na računalo. Ukoliko mobilne i stolne aplikacije komuniciraju s poslužiteljem i/ili bazom podataka, opisati i postupak njihovog postavljanja. Pri izradi uputa preporučuje se \textbf{naglasiti korake instalacije uporabom natuknica} te koristiti što je više moguće \textbf{slike ekrana} (engl. screenshots) kako bi upute bile jasne i jednostavne za slijediti.}
			
			
			 \textit{Dovršenu aplikaciju potrebno je pokrenuti na javno dostupnom poslužitelju. Studentima se preporuča korištenje neke od sljedećih besplatnih usluga: \href{https://aws.amazon.com/}{Amazon AWS}, \href{https://azure.microsoft.com/en-us/}{Microsoft Azure} ili \href{https://www.heroku.com/}{Heroku}. Mobilne aplikacije trebaju biti objavljene na F-Droid, Google Play ili Amazon App trgovini.}
			
			
			\eject 
		\chapter{Zaključak i budući rad}
		
		\textbf{\textit{dio 2. revizije}}\\
		
		 \textit{U ovom poglavlju potrebno je napisati osvrt na vrijeme izrade projektnog zadatka, koji su tehnički izazovi prepoznati, jesu li riješeni ili kako bi mogli biti riješeni, koja su znanja stečena pri izradi projekta, koja bi znanja bila posebno potrebna za brže i kvalitetnije ostvarenje projekta i koje bi bile perspektive za nastavak rada u projektnoj grupi.}
		
		 \textit{Potrebno je točno popisati funkcionalnosti koje nisu implementirane u ostvarenoj aplikaciji.}
		
		\eject 
		\chapter*{Popis literature}
		\addcontentsline{toc}{chapter}{Popis literature}
	 	
 		\textbf{\textit{Kontinuirano osvježavanje}}
	
		\textit{Popisati sve reference i literaturu koja je pomogla pri ostvarivanju projekta.}
		
		
		\begin{enumerate}
			
			
			\item  Programsko inženjerstvo, FER ZEMRIS, \url{http://www.fer.hr/predmet/proinz}
			
			\item  I. Sommerville, "Software engineering", 8th ed, Addison Wesley, 2007.
			
			\item  T.C.Lethbridge, R.Langaniere, "Object-Oriented Software Engineering", 2nd ed. McGraw-Hill, 2005.
			
			\item  I. Marsic, Software engineering book``, Department of Electrical and Computer Engineering, Rutgers University, \url{http://www.ece.rutgers.edu/~marsic/books/SE}
			
			\item  The Unified Modeling Language, \url{https://www.uml-diagrams.org/}
			
			\item  Astah Community, \url{http://astah.net/editions/uml-new}
		\end{enumerate}
		
		 
		
		
		\begingroup
		\renewcommand*\listfigurename{Indeks slika i dijagrama}
		%\renewcommand*\listtablename{Indeks tablica}
		%\let\clearpage\relax
		\listoffigures
		%\vspace{10mm}
		%\listoftables
		\endgroup
		\addcontentsline{toc}{chapter}{Indeks slika i dijagrama}
		
		
		
		\eject 
		
		%\chapter*{Dodatak: Prikaz aktivnosti grupe}
		\addcontentsline{toc}{chapter}{Dodatak: Prikaz aktivnosti grupe}
		
		\section*{Dnevnik sastajanja}
		
		\textbf{\textit{Kontinuirano osvježavanje}}\\
		
		 \textit{U ovom dijelu potrebno je redovito osvježavati dnevnik sastajanja prema predlošku.}
		
		\begin{packed_enum}
			\item  sastanak
			
			\item[] \begin{packed_item}
				\item Datum: u ovom formatu: \today
				\item Prisustvovali: I.Prezime, I.Prezime
				\item Teme sastanka:
				\begin{packed_item}
					\item  opis prve teme
					\item  opis druge teme
				\end{packed_item}
			\end{packed_item}
			
			\item  sastanak
			\item[] \begin{packed_item}
				\item Datum: u ovom formatu: \today
				\item Prisustvovali: I.Prezime, I.Prezime
				\item Teme sastanka:
				\begin{packed_item}
					\item  opis prve teme
					\item  opis druge teme
				\end{packed_item}
			\end{packed_item}
			
			%
			
		\end{packed_enum}
		
		\eject
		\section*{Tablica aktivnosti}
		
			\textbf{\textit{Kontinuirano osvježavanje}}\\
			
			 \textit{Napomena: Doprinose u aktivnostima treba navesti u satima po članovima grupe po aktivnosti.}

			\begin{longtblr}[
					label=none,
				]{
					vlines,hlines,
					width = \textwidth,
					colspec={X[7, l]X[1, c]X[1, c]X[1, c]X[1, c]X[1, c]X[1, c]X[1, c]}, 
					vline{1} = {1}{text=\clap{}},
					hline{1} = {1}{text=\clap{}},
					rowhead = 1,
				} 
			
				\SetCell[c=1]{c}{} & \SetCell[c=1]{c}{\rotatebox{90}{\textbf{Ime Prezime voditelja}}} & \SetCell[c=1]{c}{\rotatebox{90}{\textbf{Ime Prezime }}} &	\SetCell[c=1]{c}{\rotatebox{90}{\textbf{Ime Prezime }}} & \SetCell[c=1]{c}{\rotatebox{90}{\textbf{Ime Prezime }}} &	\SetCell[c=1]{c}{\rotatebox{90}{\textbf{Ime Prezime }}} & \SetCell[c=1]{c}{\rotatebox{90}{\textbf{Ime Prezime }}} &	\SetCell[c=1]{c}{\rotatebox{90}{\textbf{Ime Prezime }}} \\  
				Upravljanje projektom 		&  &  &  &  &  &  & \\ 
				Opis projektnog zadatka 	&  &  &  &  &  &  & \\ 
				
				Funkcionalni zahtjevi       &  &  &  &  &  &  &  \\ 
				Opis pojedinih obrazaca 	&  &  &  &  &  &  &  \\ 
				Dijagram obrazaca 			&  &  &  &  &  &  &  \\ 
				Sekvencijski dijagrami 		&  &  &  &  &  &  &  \\ 
				Opis ostalih zahtjeva 		&  &  &  &  &  &  &  \\ 

				Arhitektura i dizajn sustava	 &  &  &  &  &  &  &  \\ 
				Baza podataka				&  &  &  &  &  &  &   \\ 
				Dijagram razreda 			&  &  &  &  &  &  &   \\ 
				Dijagram stanja				&  &  &  &  &  &  &  \\ 
				Dijagram aktivnosti 		&  &  &  &  &  &  &  \\ 
				Dijagram komponenti			&  &  &  &  &  &  &  \\ 
				Korištene tehnologije i alati 		&  &  &  &  &  &  &  \\ 
				Ispitivanje programskog rješenja 	&  &  &  &  &  &  &  \\ 
				Dijagram razmještaja			&  &  &  &  &  &  &  \\ 
				Upute za puštanje u pogon 		&  &  &  &  &  &  &  \\  
				Dnevnik sastajanja 			&  &  &  &  &  &  &  \\ 
				Zaključak i budući rad 		&  &  &  &  &  &  &  \\  
				Popis literature 			&  &  &  &  &  &  &  \\  
				&  &  &  &  &  &  &  \\ \hline 
				\textit{Dodatne stavke kako ste podijelili izradu aplikacije} 			&  &  &  &  &  &  &  \\ 
				\textit{npr. izrada početne stranice} 				&  &  &  &  &  &  &  \\  
				\textit{izrada baze podataka} 		 			&  &  &  &  &  &  & \\  
				\textit{spajanje s bazom podataka} 							&  &  &  &  &  &  &  \\ 
				\textit{back end} 							&  &  &  &  &  &  &  \\  
				 							&  &  &  &  &  &  &\\ 
			\end{longtblr}
					
					
		\eject
		\section*{Dijagrami pregleda promjena}
		
		\textbf{\textit{dio 2. revizije}}\\
		
		\textit{Prenijeti dijagram pregleda promjena nad datotekama projekta. Potrebno je na kraju projekta generirane grafove s gitlaba prenijeti u ovo poglavlje dokumentacije. Dijagrami za vlastiti projekt se mogu preuzeti s gitlab.com stranice, u izborniku Repository, pritiskom na stavku Contributors.}
		
	
		
		
	\end{document} %naredbe i tekst nakon ove naredbe ne ulaze u izgrađen dokument 