\chapter{Specifikacija programske potpore}
		
	\section{Funkcionalni zahtjevi}
			
			\textbf{\textit{dio 1. revizije}}\\
			
			\textit{Navesti \textbf{dionike} koji imaju \textbf{interes u ovom sustavu} ili  \textbf{su nositelji odgovornosti}. To su prije svega korisnici, ali i administratori sustava, naručitelji, razvojni tim.}\\
				
			\textit{Navesti \textbf{aktore} koji izravno \textbf{koriste} ili \textbf{komuniciraju sa sustavom}. Oni mogu imati inicijatorsku ulogu, tj. započinju određene procese u sustavu ili samo sudioničku ulogu, tj. obavljaju određeni posao. Za svakog aktora navesti funkcionalne zahtjeve koji se na njega odnose.}\\
			
			
			\noindent \textbf{Dionici:}
			
			\begin{packed_enum}
				
				\item Neregistrirani korisnici
				\item Zaposlenici
				\item Revizori	
				\item Računovođe
				\item Direktor
				\item Razvojni tim
				
			\end{packed_enum}
			
			\noindent \textbf{Aktori i njihovi funkcionalni zahtjevi:}
			
			
			\begin{packed_enum}
				
				\item  \underbar{Neregistrirani korisnik (inicijator/sudionik) može:}
				\begin{packed_enum}
					
					\item Registrirati se u sustav
					
				\end{packed_enum}
			
				\item  \underbar{Zaposlenik (inicijator/sudionik) može:}
				
				\begin{packed_enum}
					
					\item Prijaviti se u sustav
					\item Učitati fotografiju u sustav te izvršiti konverziju te slike u dokument
					\item Učitati više fotografija od jednom u sustav te izvršiti njihovu konverziju u dokumente od jednom	
					\item Potvrditi ili odbiti točnost konverzije dokumenata u sustav
					\item Poslati dokument revizoru na pregled
					\item Pregledavati povijest svih dokumenata koje je zaposlenik unio u sustav
					
				\end{packed_enum}
			
				\item  \underbar{Revizor (inicijator/sudionik) može:}
				
				\begin{packed_enum}

					\item Provjeriti valjanost dokumenta kojeg mu je poslao zaposlenik
					\item Učitati fotografiju u sustav te izvršiti konverziju te slike u dokument 
					\item Učitati više fotografija od jednom u sustav te izvršiti njihovu konverziju u dokumente od jednom
					\item Potvrditi ili odbiti točnost konverzije dokumenata učitanih u sustav
					\item \textit{Aplikacija sama određuje kategoriju dokumenta no revizor to može ispraviti}
					\item Proslijediti dokument računovođi odgovornom za vrstu dokumenta kojoj dokument pripada
					\item Pregledati povijest svih dokumenata koje je revizor provjerio ili unio u sustav
					
				\end{packed_enum}
			
				\item  \underbar{Računovođa (inicijator/sudionik) može:}
				
				\begin{packed_enum}
					\item Računovođa može učitati jednu ili više slika u sustav te izvršiti njihovu konverziju u dokumente
					\item \textit{Automatsko prevođenje koje treba nekako pojasniti}
					\item Poslati dokumente direktoru na potpisivanje
					\item Arhivirati dokumente
					\item Pregledati povijest svih dokumenata one vrste za koju je računovođa zadužen
				\end{packed_enum}
			
				\item  \underbar{Direktor (inicijator/sudionik) može:}
			
				\begin{packed_enum}
					
					\item Potpisati dokument kojeg mu je poslao računovođa na potpisivanje
					\item Učitati jednu ili više fotografija u sustav te izvršiti njihovu konverziju u dokumente
					\item Potvrditi ili odbiti točnost konverzije dokumenata unesenih u sustav
					\item \textit{Automatizacija detekcije kategorije}
					\item Proslijediti dokument računovođi odgovornom za vrstu dokumenta kojoj dokument pripada
					
				\end{packed_enum}
			
			\end{packed_enum}
			
			\eject 
			
			
				
			\subsection{Obrasci uporabe}
					
				\subsubsection{Opis obrazaca uporabe}
					\textit{Funkcionalne zahtjeve razraditi u obliku obrazaca uporabe. Svaki obrazac je potrebno razraditi prema donjem predlošku. Ukoliko u nekom koraku može doći do odstupanja, potrebno je to odstupanje opisati i po mogućnosti ponuditi rješenje kojim bi se tijek obrasca vratio na osnovni tijek.}\\
					

					\noindent \underbar{\textbf{UC1 - Registracija}}
					\begin{packed_item}
	
						\item \textbf{Glavni sudionik:} Neregistrirani korisnik
						\item  \textbf{Cilj:} Registracija korisnika u sustav
						\item  \textbf{Sudionici:} Baza podataka
						\item  \textbf{Preduvjet:} -
						\item  \textbf{Opis osnovnog tijeka:}
						\item[] \begin{packed_enum}
	
							\item Neregistrirani korisnik odabire opciju za registraciju
							\item Neregistrirani korisnik unosi podatke za registraciju
							\item Korisnik prima obavijest o uspješnoj registraciji
							
						\end{packed_enum}
						
						\item  \textbf{Opis mogućih odstupanja:}
						\item[] \begin{packed_item}
	
							\item[2.a] Korisnik je odabrao već zauzetu ili neispravnu e-mail adresu, ili je dao ne dovoljno sigurnu lozinku
							\item[] \begin{packed_enum}
								
								\item Neregistriranog korisnika se vraća na stranicu za registraciju te ga se obavještava o neuspješnoj registraciji
								\item Korisnik mijenja nepravilne podatke ili odustaje od registracije	
							\end{packed_enum}
						\end{packed_item}
					\end{packed_item}
					
					
					\noindent \underbar{\textbf{UC2 - Prijava u sustav}}
					\begin{packed_item}
						
						\item \textbf{Glavni sudionik: } Korisnik sustava
						\item  \textbf{Cilj:} Prijava u sustav te pristup korisničkom sučelju
						\item  \textbf{Sudionici:} Baza podataka
						\item  \textbf{Preduvjet:}Registracija
						\item  \textbf{Opis osnovnog tijeka:}
						
						\item[] \begin{packed_enum}
							
							\item Korisnik odabire opciju za prijavu
							\item Korisnik unosi potrebne korisničke podatke
							\item Korisnik dobiva pristup korisničkom sučelju
						\end{packed_enum}
						
						\item  \textbf{Opis mogućih odstupanja:}
						
						\item[] \begin{packed_item}
							
							\item[2.a] Korisnik je upisao nepostojeće ili pogrešne korisničke podatke
							\item[] \begin{packed_enum}
								\item Korisnika se vraća na stranicu za prijavu te ga se obavještava o neuspjeloj prijavi
							\end{packed_enum}	
						\end{packed_item}
					\end{packed_item}
				
				\noindent \underbar{\textbf{UC3 -Promjena lozinke}}
				\begin{packed_item}
					
					\item \textbf{Glavni sudionik: } Korisnik
					\item  \textbf{Cilj:} Promjena lozinke koja se koristi pri prijavi u sustav
					\item  \textbf{Sudionici:} Baza podataka
					\item  \textbf{Preduvjet:} Registracija
					\item  \textbf{Opis osnovnog tijeka:}
					
					\item[] \begin{packed_enum}
						
						\item Korisnik bira opciju za promjenu lozinke
						\item Unosi staru lozinku kao potvrdu svog identiteta
						\item Bira novu lozinku te je upisuje dva puta
						\item Korisnik je obaviješten o uspješnoj promjeni lozink
					\end{packed_enum}
					
					\item  \textbf{Opis mogućih odstupanja:}
					
					\item[] \begin{packed_item}
						
						\item[2.a] Korisnik je unio pogrešnu staru lozinku
						\item[] \begin{packed_enum}
							
							\item Korisnika se obavještava o pogrešnom unosu lozinke te ga se vraća na stranicu za promjenu lozinke
							\item Korisnik ponovno unosi staru lozinku ili odustaje od promjene lozinke
							
						\end{packed_enum}
					\end{packed_item}
				\end{packed_item}
				
				
				\noindent \underbar{\textbf{UC4 - Skeniranje jedne fotografije}}
				\begin{packed_item}
					
					\item \textbf{Glavni sudionik: }Korisnik
					\item  \textbf{Cilj:} Unijeti fotografiju u sustav te izvršiti njenu konverziju u dokument
					\item  \textbf{Sudionici:} Baza podataka
					\item  \textbf{Preduvjet:} Prijava u sustav
					\item  \textbf{Opis osnovnog tijeka:}
					
					\item[] \begin{packed_enum}
						
						\item Korisnik odabire opciju za učitavanje jedne fotografije
						\item Korisnik odabire fotografiju iz datotečnog sustava svog računala
						\item Korisnik dobiva obavijest o provedenoj konverziji
						
					\end{packed_enum}
					
					\item  \textbf{Opis mogućih odstupanja:}
					
					\item[] \begin{packed_item}
						
						\item[2.a] Korisnik je odabrao datoteku koja ne postoji na njegovom računalu
						\item[] \begin{packed_enum}
							
							\item Sustav obavještava korisnika o nastaloj pogrešci te ga preusmjerava na korisničko sučelje
							
						\end{packed_enum}
					\end{packed_item}
				\end{packed_item}
				
				\noindent \underbar{\textbf{UC5 -Skeniranje više fotografija}}
				\begin{packed_item}
					
					\item \textbf{Glavni sudionik: }Korisnik
					\item  \textbf{Cilj:} Unijeti više fotografija u sustav te izvršiti njihovu konverziju u dokumente
					\item  \textbf{Sudionici:} Baza podataka
					\item  \textbf{Preduvjet:} Prijava u sustav
					\item  \textbf{Opis osnovnog tijeka:}
					
					\item[] \begin{packed_enum}
						
						\item Korisnik odabire opciju za učitavanje više dokumenata
						\item Korisnik odabire više datoteka iz svog datotečnog sustava
						\item Korisnik dobiva obavijest o provedenim konverzijama
					
					\end{packed_enum}
					
					\item  \textbf{Opis mogućih odstupanja:}
					
					\item[] \begin{packed_item}
						
						\item[2.a] Jedna od fotografija koje je korisnik unio ne postoji u datotečnom sustavu
						\item[] \begin{packed_enum}
							
							\item Sustav provodi konverziju pronađenih dokumenata
							\item Sustav šalje korisniku obavijest o ne pronađenim dokumentima
							
						\end{packed_enum}
					\end{packed_item}
				\end{packed_item}
			
			
				\noindent \underbar{\textbf{UC6 -Potvrda ispravnosti konverzije}}
				\begin{packed_item}
					
					\item \textbf{Glavni sudionik: }
					\item  \textbf{Cilj:} Potvrditi točnost konverzije dokumenta unesenih u sustav
					\item  \textbf{Sudionici:} Baza podataka
					\item  \textbf{Preduvjet:} Korisnik je prijavljen te je učitao jednu ili više fotografija u sustav
					\item  \textbf{Opis osnovnog tijeka:}
					
					\item[] \begin{packed_enum}
						
						\item Korisnik odabire opciju za potvrdu ispravnosti konverzije
						\item Korisniku se prikazuju fotografija i dokument generiran iz slike paralelno na ekranu, te mu se nudi opcija za prihvaćanje ili odbijanje točnosti konverzije
						 
						
					\end{packed_enum}
				\end{packed_item}
			
				\noindent \underbar{\textbf{UC7 -Prosljeđivanje dokumenata}}
				\begin{packed_item}
					
					\item \textbf{Glavni sudionik: }Korisnik
					\item \textbf{Cilj:} Proslijediti dokument potvrđene točnosti drugom korisniku s prikladnom razinom ovlasti za daljnju obradu dokumenta
					\item \textbf{Sudionici:} Baza podataka i drugi korisnici s višom razinom ovlasti
					\item \textbf{Preduvjet:} Korisnik je unio u sustav fotografiju, izvršio konverziju i potvrdio njenu točnost
					\item \textbf{Opis osnovnog tijeka:}
					
					\item[] \begin{packed_enum}
						
						\item Korisnik odabire opciju za slanje dokumenta
						\item Sustav prosljeđuje dokument odgovarajućem korisniku
						\item Korisnik dobiva obavijest to proslijeđenosti dokumenta
					\end{packed_enum}
				\end{packed_item}
			
				\noindent \underbar{\textbf{UC8 -Pregled vlastite povijesti}}
				\begin{packed_item}
					
					\item \textbf{Glavni sudionik: }Zaposlenik
					\item  \textbf{Cilj:} Pregled svih dokumenata koje je zaposlenik unio u sustav
					\item  \textbf{Sudionici:} Baza podataka
					\item  \textbf{Preduvjet:} Prijava
					\item  \textbf{Opis osnovnog tijeka:}
					
					\item[] \begin{packed_enum}
						
						\item Zaposlenik odabire opciju za pregled povijesti
						\item Aplikacija prikazuje korisniku listu svih dokumenata koje je unio u sustav
						\item Zaposlenik odabire dokument iz liste
						\item Aplikacija korisniku prikazuje dokument i sliku iz koje je dokument generiran
						
					\end{packed_enum}
				\end{packed_item}
			
			
		
			\noindent \underbar{\textbf{UC9 -Verifikacija točnosti konverzije}}
			\begin{packed_item}
				
				\item \textbf{Glavni sudionik: } Korisnik
				\item  \textbf{Cilj:} Potvrditi točnost generiranog dokumenta temeljem slike iz koje je dokument generiran
				\item  \textbf{Sudionici:} Baza podataka
				\item  \textbf{Preduvjet:} Skeniranje jedne ili više fotografija
				\item  \textbf{Opis osnovnog tijeka:}
				
				\item[] \begin{packed_enum}
					
					\item Korisnik odabire opciju za potvrdu konverzije
					\item Aplikacija korisniku prikazuje listu dokumenata čiju je točnost konverzije potrebno potvrditi
					\item Korisnik odabire dokument s liste, te mu aplikacija prikazuje dokument i sliku iz koje je dokument generiran. Dodatno aplikacija nudi mogućnost potvrde ili odbijanja točnosti konverzije
					\item Odabir korisnika o točnosti (ili netočnosti) konverzije pohranjuje se u bazu podataka
					
				\end{packed_enum}
			\end{packed_item}
		
			\noindent \underbar{\textbf{UC10 -Automatska kategorizacija dokumenata}}
			\begin{packed_item}
				
				\item \textbf{Glavni sudionik: }Revizor ili Računovođa ili Direktor
				\item  \textbf{Cilj:} Automatsko prepoznavanje vrste dokumenta pri konverziji dokumenta iz slike
				\item  \textbf{Sudionici:} Baza podataka
				\item  \textbf{Preduvjet:} Skeniranje jedne ili više fotografija
				\item  \textbf{Opis osnovnog tijeka:}
				
				\item[] \begin{packed_enum}
					
					\item Revizor ili Računovođa ili Direktor unose jednu ili više fotografija u sustav
					\item Sustav nakon konverzije automatski određuje kategoriju dokumenta
					\item Pri verifikaciji točnosti konverzije korisnicima se omogućuje i verifikacija točnosti automatske kategorizacije
					\item Nakon što je točnost automatske kategorizacije dokumenta potvrđena, kategorija dokumenta pohranjuje se u bazu podataka
				\end{packed_enum}
			\end{packed_item}
			
			\noindent \underbar{\textbf{UC11 -Slanje dokumenata na potpis}}
			\begin{packed_item}
				
				\item \textbf{Glavni sudionik: } Računovođa
				\item  \textbf{Cilj:} Slanje dokumenta Direktoru na potpis
				\item  \textbf{Sudionici:} Baza podataka, Direktor
				\item  \textbf{Preduvjet:} Prisutnost jednog ili više dokumenata koje treba obraditi u sustavu
				\item  \textbf{Opis osnovnog tijeka:}
				
				\item[] \begin{packed_enum}
					
					\item Računovođa bira jedan dokument iz liste dokumenata koje treba obraditi 
					\item Računovođa odabire opciju za slanje dokumenta za potpis
					\item Dokument je privremeno uklonjen iz liste dokumenata koje treba obraditi dok direktor ne potpiše dokument
				\end{packed_enum}
			\end{packed_item}
		
			\noindent \underbar{\textbf{UC12 -Arhiviranje}}
			\begin{packed_item}
				
				\item \textbf{Glavni sudionik: }Računovođa
				\item  \textbf{Cilj:} Arhiviranje dokumenata
				\item  \textbf{Sudionici:} Baza podataka
				\item  \textbf{Preduvjet:} Prisutnost jednog ili više dokumenata koji su spremni za arhiviranje
				\item  \textbf{Opis osnovnog tijeka:}
				
				\item[] \begin{packed_enum}
					
					\item Računovođa bira jedan ili više dokumenata iz liste dokumenata koje treba obraditi 
					\item Aplikacija nudi mogućnost za arhiviranje jednog ili više dokumenata
					\item Odabirom opcije za arhiviranje svi odabrani dokumenti dobivaju jedinstven broj arhiva
					\item Baza podataka pohranjuje jedinstven broj arhiva za svaki arhiviran dokument
					
				\end{packed_enum}
			\end{packed_item}
			
			\noindent \underbar{\textbf{UC$<$broj obrasca$>$ -$<$ime obrasca$>$}}
			\begin{packed_item}
				
				\item \textbf{Glavni sudionik: }$<$sudionik$>$
				\item  \textbf{Cilj:} $<$cilj$>$
				\item  \textbf{Sudionici:} $<$sudionici$>$
				\item  \textbf{Preduvjet:} $<$preduvjet$>$
				\item  \textbf{Opis osnovnog tijeka:}
				
				\item[] \begin{packed_enum}
					
					\item $<$opis korak jedan$>$
					\item $<$opis korak dva$>$
					\item $<$opis korak tri$>$
					\item $<$opis korak četiri$>$
					\item $<$opis korak pet$>$
				\end{packed_enum}
				
				\item  \textbf{Opis mogućih odstupanja:}
				
				\item[] \begin{packed_item}
					
					\item[2.a] $<$opis mogućeg scenarija odstupanja u koraku 2$>$
					\item[] \begin{packed_enum}
						
						\item $<$opis rješenja mogućeg scenarija korak 1$>$
						\item $<$opis rješenja mogućeg scenarija korak 2$>$
						
					\end{packed_enum}
					\item[2.b] $<$opis mogućeg scenarija odstupanja u koraku 2$>$
					\item[3.a] $<$opis mogućeg scenarija odstupanja  u koraku 3$>$
					
				\end{packed_item}
			\end{packed_item}
		
			\noindent \underbar{\textbf{UC13 -Pregled svih dokumenata određene kategorije}}
			\begin{packed_item}
				
				\item \textbf{Glavni sudionik: }Računovođa
				\item  \textbf{Cilj:} Pregled svih dokumenata određene kategorije koje su svi korisnici unijeli u sustav
				\item  \textbf{Sudionici:} Baza podataka
				\item  \textbf{Preduvjet:} Prisutnost jednog ili više dokumenata pretraživane kategorije u sustavu
				\item  \textbf{Opis osnovnog tijeka:}
				
				\item[] \begin{packed_enum}
					
					\item Računovođa odabire opciju za pregled povijesti svih dokumenata određene kategorije
					\item Aplikacije prikazuje računovođi listu svih dokumenata određene kategorije u sustavu
					\item Računovođa odabire dokument iz liste
					\item Aplikacija prikazuje računovođi dokument i sliku iz koje je dokument generiran
					\item $<$opis korak pet$>$
				\end{packed_enum}
			\end{packed_item}
			
			\noindent \underbar{\textbf{UC14 -Potpisivanje dokumenta}}
			\begin{packed_item}
				
				\item \textbf{Glavni sudionik: }Direktor
				\item  \textbf{Cilj:} Potpisivanje određenog dokumenta
				\item  \textbf{Sudionici:} Baza podataka
				\item  \textbf{Preduvjet:} Prisutnost jednog ili više dokumenata u sustavu za koje je zatražen potpis
				\item  \textbf{Opis osnovnog tijeka:}
				
				\item[] \begin{packed_enum}
					
					\item Direktor odabire opciju za potpisivanje dokumenata
					\item Aplikacija direktoru prikazuje listu dokumenata za koje je zatražen potpis
					\item Odabirom dokumenta aplikacija nudi direktoru opciju za potpis
					\item Direktor potpisuje dokument ili odbija potpisati dokument
				\end{packed_enum}
			\end{packed_item}
			
			\noindent \underbar{\textbf{UC15 - Promjena kategorije dokumenta}}
			\begin{packed_item}
				
				\item \textbf{Glavni sudionik: } Računovođa, Direktor, Revizor
				\item  \textbf{Cilj:} Promjena kategorije dokumenta kojeg su u sustav unijeli računovođa ili direktor
				\item  \textbf{Sudionici:} Baza podataka
				\item  \textbf{Preduvjet:} Prisutnost jednog ili više dokumenata u sustavu čija je ispravnost konverzije ispravna to automatska kategorizacija nije
				\item  \textbf{Opis osnovnog tijeka:}
				
				\item[] \begin{packed_enum}
					
					\item Direktor, računovođa ili revizor odabiru opciju za potvrdu ispravnosti automatske kategorizacije dokumenta
					\item Aplikacija prikazuje fotografija i dokument generiran iz fotografije te mu se nudi opcija za potvrdu točnosti ili promjenu kategorije dokumenta
					\item Baza podataka pohranjuje odabir korisnika
				\end{packed_enum}
			\end{packed_item}
			
			\noindent \underbar{\textbf{UC16 -Objava dokumenata na društvenoj mreži}}
			\begin{packed_item}
				
				\item \textbf{Glavni sudionik: } Direktor
				\item  \textbf{Cilj:} Objava dokumenata na društvenoj mreži
				\item  \textbf{Sudionici:} Baza podataka
				\item  \textbf{Preduvjet:} Prisutnost dokumenata u sustavu
				\item  \textbf{Opis osnovnog tijeka:}
				
				\item[] \begin{packed_enum}
					
					\item Direktor odabire opciju za objavu dokumenta na društvenoj mreži
					\item Direktor odabire na kojoj od ponuđenih društvenih mreža želi objaviti dokument
					\item Direktor unosi potrebne podatke za objavu dokumenta koje traži specifična društvena mreža
					\item Dokument se objavljuje na društvenoj mreži
					
				\end{packed_enum}
			\end{packed_item}
		
			\noindent \underbar{\textbf{UC17 -Pregled povijesti svih dokumenata}}
			\begin{packed_item}
				
				\item \textbf{Glavni sudionik: }Direktor
				\item  \textbf{Cilj:} Pregled povijesti svih dokumenata
				\item  \textbf{Sudionici:} Baza podataka
				\item  \textbf{Preduvjet:} Prisutnost dokumenata u sustavu
				\item  \textbf{Opis osnovnog tijeka:}
				
				\item[] \begin{packed_enum}
					
					\item Direktor odabire opciju za pregled povijesti dokumenata
					\item Aplikacija direktoru prikazuje listu svih dokumenata ikad unesenih u sustav
					\item Odabirom dokumenta sa liste prikazuje se fotografija i dokument generiran iz fotografije
				
				\end{packed_enum}
			\end{packed_item}
		
			\noindent \underbar{\textbf{UC18 -Brisanje dokumenta iz Arhiva}}
			\begin{packed_item}
				
				\item \textbf{Glavni sudionik: }Direktor
				\item  \textbf{Cilj:}Brisanje dokumenta iz Arhiva
				\item  \textbf{Sudionici:} Baza podataka
				\item  \textbf{Preduvjet:} Prisutnost jednog ili više arhiviranih dokumenata u sustavu
				\item  \textbf{Opis osnovnog tijeka:}
				
				\item[] \begin{packed_enum}
					
					\item Direktor odabire opciju za brisanje dokumenta iz arhiva
					\item Aplikacija prikazuje listu svih arhiviranih dokumenata, te direktor odabire onog kojeg želi učitati
					\item Direktor unosi svoju lozinku kao potvrdu svog identiteta i odluke
					\item Baza briše dokument iz sustava
					
				\end{packed_enum}
			\end{packed_item}
		
			\noindent \underbar{\textbf{UC19 -Pregled statistike zaposlenika}}
			\begin{packed_item}
				
				\item \textbf{Glavni sudionik: } Direktor 
				\item  \textbf{Cilj:} Pregled statistike zaposlenika
				\item  \textbf{Sudionici:} Baza podataka, korisnici
				\item  \textbf{Preduvjet:} -
				\item  \textbf{Opis osnovnog tijeka:}
				
				\item[] \begin{packed_enum}
					
					\item Direktor odabire opciju za pregled statistike zaposlenika
					\item Aplikacija nudi listu svih korisnika
					\item Direktor odabire zaposlenika čije statistike želi pogledati
					\item Aplikacija prikazuje tražene podatke
				\end{packed_enum}
			\end{packed_item}
			
			\noindent \underbar{\textbf{UC20 -Promjena razine ovlasti}}
			\begin{packed_item}
				
				\item \textbf{Glavni sudionik: }Direktor 
				\item  \textbf{Cilj:} Promjena razine ovlasti korisnika
				\item  \textbf{Sudionici:} Baza podataka, korisnici
				\item  \textbf{Preduvjet:} -
				\item  \textbf{Opis osnovnog tijeka:}
				
				\item[] \begin{packed_enum}
					
					\item Direktor odabire opciju za promjenu razine ovlasti korisnika
					\item Sustav prikazuje direktoru listu svih korisnika sustava
					\item Direktor odabire zaposlenika te mu dodjeljuje novu razinu ovlasti
					\item Baza podataka sprema unesene promjene
					
				\end{packed_enum}
			\end{packed_item}
			
			\noindent \underbar{\textbf{UC21 -Brisanje korisničkog računa}}
			\begin{packed_item}
				
				\item \textbf{Glavni sudionik: } Korisnik
				\item  \textbf{Cilj:} Brisanje korisničkog računa
				\item  \textbf{Sudionici:} Baza podataka
				\item  \textbf{Preduvjet:} Registracija
				\item  \textbf{Opis osnovnog tijeka:}
				
				\item[] \begin{packed_enum}
					
					\item Korisnik bira opciju za brisanje korisničkog računa
					\item Korisnik unosi svoju korisničku lozinku kao potvrdu svog odabira i identiteta
					\item Aplikacija korisnika preusmjerava na stranicu za prijavu
					\item Baza podataka briše korisnički račun
				\end{packed_enum}
			\end{packed_item}
				
					
				\subsubsection{Dijagrami obrazaca uporabe}
					
					\textit{Prikazati odnos aktora i obrazaca uporabe odgovarajućim UML dijagramom. Nije nužno nacrtati sve na jednom dijagramu. Modelirati po razinama apstrakcije i skupovima srodnih funkcionalnosti.}
				\eject		
				
			\subsection{Sekvencijski dijagrami}
				
				\textbf{\textit{dio 1. revizije}}\\
				
				\textit{Nacrtati sekvencijske dijagrame koji modeliraju najvažnije dijelove sustava (max. 4 dijagrama). Ukoliko postoji nedoumica oko odabira, razjasniti s asistentom. Uz svaki dijagram napisati detaljni opis dijagrama.}
				\eject
	
		\section{Ostali zahtjevi}
		
			\textbf{\textit{dio 1. revizije}}\\
		 
			 \textit{Nefunkcionalni zahtjevi i zahtjevi domene primjene dopunjuju funkcionalne zahtjeve. Oni opisuju \textbf{kako se sustav treba ponašati} i koja \textbf{ograničenja} treba poštivati (performanse, korisničko iskustvo, pouzdanost, standardi kvalitete, sigurnost...). Primjeri takvih zahtjeva u Vašem projektu mogu biti: podržani jezici korisničkog sučelja, vrijeme odziva, najveći mogući podržani broj korisnika, podržane web/mobilne platforme, razina zaštite (protokoli komunikacije, kriptiranje...)... Svaki takav zahtjev potrebno je navesti u jednoj ili dvije rečenice.}
			 
			 
			 
	