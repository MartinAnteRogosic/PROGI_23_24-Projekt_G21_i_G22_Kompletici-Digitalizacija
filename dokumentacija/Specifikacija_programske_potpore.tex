\chapter{Specifikacija programske potpore}
		
	\section{Funkcionalni zahtjevi}
			
			\textbf{\textit{dio 1. revizije}}\\
			
			\textit{Navesti \textbf{dionike} koji imaju \textbf{interes u ovom sustavu} ili  \textbf{su nositelji odgovornosti}. To su prije svega korisnici, ali i administratori sustava, naručitelji, razvojni tim.}\\
				
			\textit{Navesti \textbf{aktore} koji izravno \textbf{koriste} ili \textbf{komuniciraju sa sustavom}. Oni mogu imati inicijatorsku ulogu, tj. započinju određene procese u sustavu ili samo sudioničku ulogu, tj. obavljaju određeni posao. Za svakog aktora navesti funkcionalne zahtjeve koji se na njega odnose.}\\
			
			
			\noindent \textbf{Dionici:}
			
			\begin{packed_enum}
				
				\item Zaposlenici
				\item Revizori	
				\item Računovođe
				\item Direktor
				\item Razvojni tim
				
			\end{packed_enum}
			
			\noindent \textbf{Aktori i njihovi funkcionalni zahtjevi:}
			
			
			\begin{packed_enum}
				\item  \underbar{Zaposlenik (inicijator/sudionik) može:}
				
				\begin{packed_enum}
					
					\item Prijaviti se u sustav
					\item Učitati fotografiju u sustav te izvršiti konverziju te slike u dokument
					\item Učitati više fotografija od jednom u sustav te izvršiti njihovu konverziju u dokumente od jednom	
					\item Potvrditi ili odbiti točnost konverzije dokumenata u sustav
					\item Poslati dokument revizoru na pregled
					\item Pregledavati povijest svih dokumenata koje je zaposlenik unio u sustav
					
				\end{packed_enum}
			
				\item  \underbar{Revizor (inicijator/sudionik) može:}
				
				\begin{packed_enum}

					\item Provjeriti valjanost dokumenta kojeg mu je poslao zaposlenik
					\item Učitati fotografiju u sustav te izvršiti konverziju te slike u dokument 
					\item Učitati više fotografija od jednom u sustav te izvršiti njihovu konverziju u dokumente od jednom
					\item Potvrditi ili odbiti točnost konverzije dokumenata učitanih u sustav
					\item \textit{Aplikacija sama određuje kategoriju dokumenta no revizor to može ispraviti}
					\item Proslijediti dokument računovođi odgovornom za vrstu dokumenta kojoj dokument pripada
					\item Pregledati povijest svih dokumenata koje je revizor provjerio ili unio u sustav
					
				\end{packed_enum}
			
				\item  \underbar{Računovođa (inicijator/sudionik) može:}
				
				\begin{packed_enum}
					\item Računovođa može učitati jednu ili više slika u sustav te izvršiti njihovu konverziju u dokumente
					\item \textit{Automatsko prevođenje koje treba nekako pojasniti}
					\item Poslati dokumente direktoru na potpisivanje
					\item Arhivirati dokumente
					\item Pregledati povijest svih dokumenata one vrste za koju je računovođa zadužen
				\end{packed_enum}
			
				\item  \underbar{Direktor (inicijator/sudionik) može:}
			
				\begin{packed_enum}
					
					\item Potpisati dokument kojeg mu je poslao računovođa na potpisivanje
					\item Učitati jednu ili više fotografija u sustav te izvršiti njihovu konverziju u dokumente
					\item Potvrditi ili odbiti točnost konverzije dokumenata unesenih u sustav
					\item \textit{Automatizacija detekcije kategorije}
					\item Proslijediti dokument računovođi odgovornom za vrstu dokumenta kojoj dokument pripada
					
				\end{packed_enum}
			
			\end{packed_enum}
			
			\eject 
			
			
				
			\subsection{Obrasci uporabe}
					
				\subsubsection{Opis obrazaca uporabe}
					\textit{Funkcionalne zahtjeve razraditi u obliku obrazaca uporabe. Svaki obrazac je potrebno razraditi prema donjem predlošku. Ukoliko u nekom koraku može doći do odstupanja, potrebno je to odstupanje opisati i po mogućnosti ponuditi rješenje kojim bi se tijek obrasca vratio na osnovni tijek.}\\
					

					\noindent \underbar{\textbf{UC$<$broj obrasca$>$ -$<$ime obrasca$>$}}
					\begin{packed_item}
	
						\item \textbf{Glavni sudionik: }$<$sudionik$>$
						\item  \textbf{Cilj:} $<$cilj$>$
						\item  \textbf{Sudionici:} $<$sudionici$>$
						\item  \textbf{Preduvjet:} $<$preduvjet$>$
						\item  \textbf{Opis osnovnog tijeka:}
						
						\item[] \begin{packed_enum}
	
							\item $<$opis korak jedan$>$
							\item $<$opis korak dva$>$
							\item $<$opis korak tri$>$
							\item $<$opis korak četiri$>$
							\item $<$opis korak pet$>$
						\end{packed_enum}
						
						\item  \textbf{Opis mogućih odstupanja:}
						
						\item[] \begin{packed_item}
	
							\item[2.a] $<$opis mogućeg scenarija odstupanja u koraku 2$>$
							\item[] \begin{packed_enum}
								
								\item $<$opis rješenja mogućeg scenarija korak 1$>$
								\item $<$opis rješenja mogućeg scenarija korak 2$>$
								
							\end{packed_enum}
							\item[2.b] $<$opis mogućeg scenarija odstupanja u koraku 2$>$
							\item[3.a] $<$opis mogućeg scenarija odstupanja  u koraku 3$>$
							
						\end{packed_item}
					\end{packed_item}
				
					
				\subsubsection{Dijagrami obrazaca uporabe}
					
					\textit{Prikazati odnos aktora i obrazaca uporabe odgovarajućim UML dijagramom. Nije nužno nacrtati sve na jednom dijagramu. Modelirati po razinama apstrakcije i skupovima srodnih funkcionalnosti.}
				\eject		
				
			\subsection{Sekvencijski dijagrami}
				
				\textbf{\textit{dio 1. revizije}}\\
				
				\textit{Nacrtati sekvencijske dijagrame koji modeliraju najvažnije dijelove sustava (max. 4 dijagrama). Ukoliko postoji nedoumica oko odabira, razjasniti s asistentom. Uz svaki dijagram napisati detaljni opis dijagrama.}
				\eject
	
		\section{Ostali zahtjevi}
		
			\textbf{\textit{dio 1. revizije}}\\
		 
			 \textit{Nefunkcionalni zahtjevi i zahtjevi domene primjene dopunjuju funkcionalne zahtjeve. Oni opisuju \textbf{kako se sustav treba ponašati} i koja \textbf{ograničenja} treba poštivati (performanse, korisničko iskustvo, pouzdanost, standardi kvalitete, sigurnost...). Primjeri takvih zahtjeva u Vašem projektu mogu biti: podržani jezici korisničkog sučelja, vrijeme odziva, najveći mogući podržani broj korisnika, podržane web/mobilne platforme, razina zaštite (protokoli komunikacije, kriptiranje...)... Svaki takav zahtjev potrebno je navesti u jednoj ili dvije rečenice.}
			 
			 
			 
	