\chapter{Zaključak i budući rad}
		
		\textbf{\textit{dio 2. revizije}}\\
		
		 \textit{U ovom poglavlju potrebno je napisati osvrt na vrijeme izrade projektnog zadatka, koji su tehnički izazovi prepoznati, jesu li riješeni ili kako bi mogli biti riješeni, koja su znanja stečena pri izradi projekta, koja bi znanja bila posebno potrebna za brže i kvalitetnije ostvarenje projekta i koje bi bile perspektive za nastavak rada u projektnoj grupi.}
		
		 \textit{Potrebno je točno popisati funkcionalnosti koje nisu implementirane u ostvarenoj aplikaciji.}
		 
		 	Zadatak projektnog tima bio je razvoj aplikacije za skeniranje i
		 distribuciju dokumenata unutar organizacije. Nakon dva i pol mjeseca rada ostvarili smo zadani cilj. \newline
		 	
			Projekt smo proveli unutar dvije faze razvoja. Prva faza uključivala
		je pisanje početne dokumentacije, odabir tehnologija te podjelu zadataka unutar projektnog tima. Prva faza je također uključivala implementaciju sustava registracije i prijave korisnika te kostur aplikacije. Druga faza sastojala se od implementacije ključnih djelova sustava kao što su skeniranje dokumenata, njihova distribucija i arhiviranje. Dodatno druga faza sadržavala je pisanje popratne dokumentacije.\newline
		 	
		 	Općenito govoreći rad na projektu bio je zanimljivo i korisno iskustvo
		 za sve članove tima. Suočili smo se sa mnogim izazovima od kojih smo mnoge riješili. Glavne prepreke su bile manjak iskustva te ne optimalna koordinacija članova tima. Manjak iskustva stvarao je poteškoće jer su članovi tima morali samostalno učiti tehnologije koje prije nisu koristili te nisu znali najbolje prakse u mnogim situacijama. Ne optimalna koordinacija članova tima značila je da količina posla nije bila u potpunosti ravnopravna te svi članovi tima nisu u potpunosti mogli pridonijeti čak i kad je postojala volja. Poboljšanje radnog procesa u budućim nadogradnjama aplikacije ili radu na novim projektima uključivala bi korištenje poznatih alata i bolju koordinaciju projektnog tima. \newline
		 
		 	Implementirali smo sve ključne značajke zadatka. Neka od poboljšanja
		 koja bi se mogla implementirati su razvoj mobilne aplikacija, poboljšanje robusnosti OCR-a i razvoj boljeg korisničkog sučelja. Iako aplikacija ima mnoge nedostatke zadovoljni smo s odrađenim poslom
		 	 
		 	
		
		\eject 